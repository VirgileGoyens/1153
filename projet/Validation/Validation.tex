\documentclass{article}

\usepackage[utf8]{inputenc}
\usepackage[T1]{fontenc}      
\usepackage[francais]{babel}
\usepackage{graphicx}
\usepackage{circuitikz}
\usepackage[squaren, Gray]{SIunits}
\usepackage{sistyle}
\usepackage[autolanguage]{numprint}
\usepackage{pgfplots}
\pgfplotsset{compat=1.9}
\usepackage{amsmath,amssymb,array}
\usepackage[top=2.5cm,bottom=2.5cm,right=2.5cm,left=2.5cm]{geometry}
\usepackage{url} 
\usepackage{tabularx}
\DeclareMathOperator{\dist}{d}
\newenvironment{abstract-fr}
{
	\begin{center}
		\textbf{Résumé} \\[0.5cm]
	\end{center}
}
{}

\newenvironment{abstract-en}
{
	\begin{center}
		\textbf{Summary} \\[0.5cm]
	\end{center}
}
{}
% New command pour la modélisation mécanique, tri à effectuer
\newcommand\fv[1]{{\bf #1}} % free vector
\newcommand\fvd[1]{\dot{\bf #1}} % free vector derivated
\newcommand\fvdd[1]{\ddot{\bf #1}} % free vector derivated
\newcommand\fvr[1]{\mathring{\bf #1}} % free vector relatively derivated
\newcommand\fvrr[1]{\overset{\circ\circ}{\bf #1}} % free vector relatively derivated
\newcommand\uv[1]{{\bf\hat{ #1}}} % unit vector
\newcommand\ui{{\bf\hat{I}}} % unit vector I
\newcommand\uj{{\bf\hat{J}}} % unit vector J
\newcommand\uk{{\bf\hat{K}}} % unit vector K
\newcommand\wrt[2]{\ensuremath{\tensor*[_{ #1}]{ #2}{}}} % With Respect To
\newcommand\wtr[3]{\ensuremath{\tensor*[_{ #1}]{ #2}{^{ #3}}}} % With Two Respect
\newcommand\omegaf{{\bm \omega}}
\newcommand\omegafr{\mathring{\bm \omega}}
\newcommand\omegafd{\dot{\bm \omega}}
\newcommand\omegaft{\tilde{\bm \omega}}
\newcommand\omegaftr{\mathring{\tilde{\bm \omega}}}
\newcommand\omegat{\tilde{\omega}}
\newcommand\omegatd{\tilde{\dot{\omega}}}
\newcommand\ine{{\bf I}}
\newcommand\st{{\bf L}}
\newcommand\pst{{\bf M}}
\newcommand\lm{{\bf N}}
\newcommand\am{{\bf H}}
\newcommand\amd{\dot{\am}}
\newcommand\fo{{\bf F}}
\newcommand\po{\mathcal{P}}
\newcommand\xg{\ensuremath{\fv{R}}}
\newcommand\xgd{\ensuremath{\fvd{R}}}
\newcommand\xgdd{\ensuremath{\fvdd{R}}}
\newcommand\dvec[1]{\dot{\vec{ #1}}}
\newcommand\ddvec[1]{\ddot{\vec{ #1}}}
\newcommand\qp{\dot{q}}
\newcommand\dqp{\Delta \dot{q}}
\usepackage{url} 
\usepackage{hyperref}
\hypersetup{
    colorlinks,
    citecolor=black,
    filecolor=black,
    linkcolor=black,
    urlcolor=black
}

\begin{document}

%La description de l’appareillage de mesure est claire et complète 
Pour pouvoir tester le circuit, nous avons arrangé les appareils comme cela: Nous relions deux générateurs fournissant +15 et 
-15 volts à la plaquette.  Elle même est reliée à la bobine mobile.  La bobine fixe a quand a elle une source de tension qui 
lui fournit un amperage de 1 A pour produire un champs magnétique suffisant.


Concernant les instruments de mesures, on place la pointe de l'oscilloscope sur la sortie de la plaquette et le fréquence
de la musique apparait sur l'écran.  On peut aussi tester le champs électrique passant dans la bobine fixe, on utilise le
teslamètre en mettant le capteur dans l'entrefer, entre la bobine et une barre du 'E'.  On se sert aussi d'un multimètre
pour mesurer la resistance des bobines et du courant qui les traverse.


Nous avons fait de notre mieux pour avoir une aussi bonne précision que possible mais nous savons qu'elle peut être grandement
améliorée, quand nous tenons le teslamètre, il est extrèment dur d'obtenir un champs continue, celui-çi varie légerement donc
nous avons pris la valeure la plus centrale mais celle-ci n'est pas très précise.


%La méthode des mesures est expliquée
A chaque fois que nous faisions un test, nous veillions à ne pas faire bouger ni l'appareil de mesure ni le bloc testé.

Nous avons fait des mesures de voltage en fonction de différentes fréquences pour les filtres RC: ce qui nous a permis
de trouver les fréquences de coupures. A l'aide de l'oscilloscope et d'un générateur, nous pouvions faire varier
la fréquence et sur l'écran de l'oscillo, il nous était possible de mesurer.  Nous avons fait ces tests avec les filtres passe haut et passe bas.  Cela correspond 
au bloc 1 et 2.

Pour la bobine fixe, nous voulions savoir quel était le champs produit par les spires, pour cela nous avons fait passer
du courant dans les spires en reliant un générateur avec les fils de la bobine et nous avons mis le capteur du teslamètre 
dans l'entrefer.  Après, nous lisions la valeur du champs sur l'écran de l'appareil.
Quand nous mesurions la resistance des bobines, nous isolions la bobine que nous voulions tester, nous la branchions au multimètre
et on regardait la valeur, nous veillons à chaque fois à bien avoir le bon ordre de grandeur. Cela renvoit au bloc 3.

Enfin, nous avons fait des tests sur la sortie de la plaquette, en testant avec l'oscilloscope.  Nous regardions l'écran de
l'appareil pour voir si une fréquence possible sortait de la plaquette.  Cela correspond au bloc 4.

%Des tests paramétriques sont effectués


%Mesures
Pour le filtre passe-bas:
\begin{center}
\begin{tabular}{|c|c|c|}
\hline
$V_c[V]$ & $f[Hz]$ & $\log{f}$ \\
\hline
1.7 & 16000 & 4.204 \\
\hline
1.55 & 18000 & 4.255 \\
\hline
1.45 & 20000 & 4.301 \\
\hline
\end{tabular}
\end{center}

Pour le filtre passe-haut

\begin{center}
	\begin{tabular}{|c|c|c|}
		\hline
		$V_c[V]$ & $f[Hz]$ & $\log{f}$ \\
		\hline
		127 & 0.4 & 2.1\\
		\hline
		191 & 0.5 & 2.3\\
		\hline
		356 & 0.6 & 2.6 \\
		\hline
	\end{tabular}
\end{center}

\begin{center}
	\begin{tabular}{|c|c|c|}
		\hline
		Resistance bobine fixe[\ohm] & Champ magn.[T] & Amperage[A] \\
		\hline
		2.38 & 0.08 & 0.1667\\
		\hline
	\end{tabular}
\end{center}


% Just here to fix rapport_prejury.tex
\end{document}
