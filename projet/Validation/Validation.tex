\documentclass{article}

\usepackage[utf8]{inputenc}
\usepackage[T1]{fontenc}      
\usepackage[francais]{babel}
\usepackage{graphicx}
\usepackage{circuitikz}
\usepackage[squaren, Gray]{SIunits}
\usepackage{sistyle}
\usepackage[autolanguage]{numprint}
\usepackage{pgfplots}
\pgfplotsset{compat=1.9}
\usepackage{amsmath,amssymb,array}
\usepackage[top=2.5cm,bottom=2.5cm,right=2.5cm,left=2.5cm]{geometry}
\usepackage{url} 
\usepackage{tabularx}
\DeclareMathOperator{\dist}{d}
\newenvironment{abstract-fr}
{
	\begin{center}
		\textbf{Résumé} \\[0.5cm]
	\end{center}
}
{}

\newenvironment{abstract-en}
{
	\begin{center}
		\textbf{Summary} \\[0.5cm]
	\end{center}
}
{}
% New command pour la modélisation mécanique, tri à effectuer
\newcommand\fv[1]{{\bf #1}} % free vector
\newcommand\fvd[1]{\dot{\bf #1}} % free vector derivated
\newcommand\fvdd[1]{\ddot{\bf #1}} % free vector derivated
\newcommand\fvr[1]{\mathring{\bf #1}} % free vector relatively derivated
\newcommand\fvrr[1]{\overset{\circ\circ}{\bf #1}} % free vector relatively derivated
\newcommand\uv[1]{{\bf\hat{ #1}}} % unit vector
\newcommand\ui{{\bf\hat{I}}} % unit vector I
\newcommand\uj{{\bf\hat{J}}} % unit vector J
\newcommand\uk{{\bf\hat{K}}} % unit vector K
\newcommand\wrt[2]{\ensuremath{\tensor*[_{ #1}]{ #2}{}}} % With Respect To
\newcommand\wtr[3]{\ensuremath{\tensor*[_{ #1}]{ #2}{^{ #3}}}} % With Two Respect
\newcommand\omegaf{{\bm \omega}}
\newcommand\omegafr{\mathring{\bm \omega}}
\newcommand\omegafd{\dot{\bm \omega}}
\newcommand\omegaft{\tilde{\bm \omega}}
\newcommand\omegaftr{\mathring{\tilde{\bm \omega}}}
\newcommand\omegat{\tilde{\omega}}
\newcommand\omegatd{\tilde{\dot{\omega}}}
\newcommand\ine{{\bf I}}
\newcommand\st{{\bf L}}
\newcommand\pst{{\bf M}}
\newcommand\lm{{\bf N}}
\newcommand\am{{\bf H}}
\newcommand\amd{\dot{\am}}
\newcommand\fo{{\bf F}}
\newcommand\po{\mathcal{P}}
\newcommand\xg{\ensuremath{\fv{R}}}
\newcommand\xgd{\ensuremath{\fvd{R}}}
\newcommand\xgdd{\ensuremath{\fvdd{R}}}
\newcommand\dvec[1]{\dot{\vec{ #1}}}
\newcommand\ddvec[1]{\ddot{\vec{ #1}}}
\newcommand\qp{\dot{q}}
\newcommand\dqp{\Delta \dot{q}}
\usepackage{url} 
\usepackage{hyperref}
\hypersetup{
    colorlinks,
    citecolor=black,
    filecolor=black,
    linkcolor=black,
    urlcolor=black
}

\begin{document}

% La description de l’appareillage de mesure est claire et complète
\section{Description de l'appareillage}
Afin de tester le circuit de notre haut-parleur, nous avions besoin d'une alimentation
capable de fournir du $\unit{\pm 15}{\volt}$ (pour alimenter le dual ampli-op ainsi
que l'amplificateur audio). Nous avions donc besoin de deux alimentations DC Topward 
\numprint{3303} DS présentes au laboratoire. Pour obtenir une tension négative par 
rapport à une tension de référence, il faut connecter ensemble les deux chassis,
et les bornes $+$ de la première source, et $-$ de la deuxième source. La première règle
alors la tension négative et la deuxième la tension positive\cite{dctopward}.
Au niveau de l'intensité du courant, étant donné que la puissance maximale de l'amplificateur
audio est de \unit{2.5}{\watt}\cite{datasheetampli} et que la tension d'alimentation est de \unit{15}{\volt}, par
la formule $P = VI$ en utilisant la valeur RMS, on obient \unit{0.23}{\ampere}. Nous règlons
donc les deux sources sur cette valeur. Les câbles des deux alimentations sont ensuite connectés 
au bornier de la plaquette. La sortie de la plaquette est quant à elle reliée à la bobine mobile.


L'électroaimant est lui alimenté en courant continu par une troisième source qui 
fourni un courant de \unit{1}{\ampere}. Le courant maximum supportable par les fils de 
cuivre dont nous disposons étant de \unit{2}{\ampere}\cite{norme-cuivre}, nous prenons
une bonne marge de sécurité afin d'éviter la surchauffe des fils.

Enfin, pour les différentes mesures, nous utilisons :

\begin{itemize}
	\item Un oscilloscope lorsque nous désirons visualiser le signal passant par un point de notre circuit ;
	\item Un multimètre pour mesurer un courant, une tension ou une résistane ;
	\item Un teslamètre pour mesurer l'intensité du champ magnétique produit par notre électroaimant.
\end{itemize}

\section{Compte-rendu des mesures}

\subsection{Mesures du champ magnétique produit par l'électroaimant}
Pour mesurer le champ magnétique produit par l'électroaimant, nous nous servons du teslamètre.
Le champ qui nous intéresse se situe dans l'entrefer de l'électroaimant, il suffit alors de placer
la sonde du teslamètre dans cet entrefer. Il est difficile d'obtenir une mesure très précise en se
servant du teslamètre car la valeur qu'il nous indique n'est pas très stable. Cependant, en répétant
plusieurs fois l'expérience et en notant les valeurs affichées par le teslamètre, nous obtenons une mesure
moyenne de \unit{8}{\centi\tesla} ; un résultat qui confirme nos prédictions théoriques
qui annonçaient un champ magnétique de \unit{7.54}{\centi\tesla}.

\subsection{Mesures en différents points du circuit}
Pour mesurer la tension du signal audio en différents points du circuit, nous utilisons l'oscilloscope
et sa sonde.
Nous avons commencé par mesurer la tension du signal audio sortant de la prise jack (cfr Figure \ref{jack}). Ensuite nous avons
placé la sonde de l'oscilloscope au point IN1 du circuit. Nous avons alors pu visualiser le
signal audio, comme le montre la Figure \ref{in1}.
En prenant une échelle appropriée, c'est-à-dire \unit{50}{\milli\volt}, on voit que le signal
audio a une amplitude de \unit{0.0375}{\volt} à la sortie de la prise jack (cfr Figure \ref{jack}).
Au point de sortie, nous avons mesuré une tension située entre \unit{3}{\volt}
et \unit{0.22}{\volt} selon que le potentiomètre lié au réglage du volume
soit sur sa valeur maximale ou sur sa valeur minimale (cfr Figure \ref{out}).


\begin{figure}[h]
\begin{minipage}[c]{.45\linewidth}
\begin{center}
\includegraphics[scale=0.10]{jack}
\caption{Signal audio visualisé à la sortie de la prise jack. Échelle: \unit{50}{\milli\volt}}
\label{jack}
\end{center}
\end{minipage}
\hfill
\begin{minipage}[c]{.45\linewidth}
\begin{center}
\includegraphics[scale=0.10]{IN1}
\caption{Signal audio visualisé au point IN1 du circuit. Échelle: \unit{50}{\milli\volt}}
\label{in1}
\end{center}
\end{minipage}
\end{figure}


\begin{figure}[h]
\begin{minipage}[c]{.45\linewidth}
\begin{center}
\includegraphics[scale=0.10]{TP2}
\caption{Signal audio visualisé au point TP2 du circuit. Échelle: \unit{0.1}{\volt}}
\label{TP2}
\end{center}
\end{minipage}
\hfill
\begin{minipage}[c]{.45\linewidth}
\begin{center}
\includegraphics[scale=0.10]{OUT}
\caption{Signal audio visualisé au point OUT du circuit. Échelle: \unit{0.1}{\volt}}
\label{out}
\end{center}
\end{minipage}
\end{figure}



% Just here to fix rapport_prejury.tex
\end{document}
