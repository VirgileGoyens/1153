\documentclass{article}

\usepackage[utf8]{inputenc}
\usepackage[T1]{fontenc}      
\usepackage[francais]{babel}
\usepackage{graphicx}
\usepackage{circuitikz}
\usepackage[squaren, Gray]{SIunits}
\usepackage{sistyle}
\usepackage[autolanguage]{numprint}
\usepackage{pgfplots}
\pgfplotsset{compat=1.9}
\usepackage{amsmath,amssymb,array}
\usepackage[top=2.5cm,bottom=2.5cm,right=2.5cm,left=2.5cm]{geometry}
\usepackage{url} 
\usepackage{tabularx}
\DeclareMathOperator{\dist}{d}
\newenvironment{abstract-fr}
{
	\begin{center}
		\textbf{Résumé} \\[0.5cm]
	\end{center}
}
{}

\newenvironment{abstract-en}
{
	\begin{center}
		\textbf{Summary} \\[0.5cm]
	\end{center}
}
{}
% New command pour la modélisation mécanique, tri à effectuer
\newcommand\fv[1]{{\bf #1}} % free vector
\newcommand\fvd[1]{\dot{\bf #1}} % free vector derivated
\newcommand\fvdd[1]{\ddot{\bf #1}} % free vector derivated
\newcommand\fvr[1]{\mathring{\bf #1}} % free vector relatively derivated
\newcommand\fvrr[1]{\overset{\circ\circ}{\bf #1}} % free vector relatively derivated
\newcommand\uv[1]{{\bf\hat{ #1}}} % unit vector
\newcommand\ui{{\bf\hat{I}}} % unit vector I
\newcommand\uj{{\bf\hat{J}}} % unit vector J
\newcommand\uk{{\bf\hat{K}}} % unit vector K
\newcommand\wrt[2]{\ensuremath{\tensor*[_{ #1}]{ #2}{}}} % With Respect To
\newcommand\wtr[3]{\ensuremath{\tensor*[_{ #1}]{ #2}{^{ #3}}}} % With Two Respect
\newcommand\omegaf{{\bm \omega}}
\newcommand\omegafr{\mathring{\bm \omega}}
\newcommand\omegafd{\dot{\bm \omega}}
\newcommand\omegaft{\tilde{\bm \omega}}
\newcommand\omegaftr{\mathring{\tilde{\bm \omega}}}
\newcommand\omegat{\tilde{\omega}}
\newcommand\omegatd{\tilde{\dot{\omega}}}
\newcommand\ine{{\bf I}}
\newcommand\st{{\bf L}}
\newcommand\pst{{\bf M}}
\newcommand\lm{{\bf N}}
\newcommand\am{{\bf H}}
\newcommand\amd{\dot{\am}}
\newcommand\fo{{\bf F}}
\newcommand\po{\mathcal{P}}
\newcommand\xg{\ensuremath{\fv{R}}}
\newcommand\xgd{\ensuremath{\fvd{R}}}
\newcommand\xgdd{\ensuremath{\fvdd{R}}}
\newcommand\dvec[1]{\dot{\vec{ #1}}}
\newcommand\ddvec[1]{\ddot{\vec{ #1}}}
\newcommand\qp{\dot{q}}
\newcommand\dqp{\Delta \dot{q}}
\usepackage{url} 
\usepackage{hyperref}
\hypersetup{
    colorlinks,
    citecolor=black,
    filecolor=black,
    linkcolor=black,
    urlcolor=black
}

\begin{document}

\section{Planning}

\subsection{Vue d'ensemble}
Au début du quadrimestre, il nous a été demandé de concevoir, réaliser et quantifier un système de haut-parleur.
Notre travail a été réparti en trois parties : les séances tutorées, les laboratoires, et le travail autonome. 
Voici une vue d'ensemble de notre planning, semaine après semaine. 

\begin{table}[!htb]
	\centering
	\begin{tabular}{|c|p{9cm}|}
		\hline 
		Semaine & Tâches \\
		\hline
			S1 & Présentation du projet et familiarisation avec les appareils de laboratoire \\
			S2 & Représentation d'un haut-parleur sous forme de blocs fonctionnels + cahier des charges \\
			S3 & Modélisation des filtres passe-haut et passe-bas + analyse complète du circuit imprimé \\
			S4 & Choix des mots clefs pour l'initiation aux recherches universitaires. Dimensionnement de l'électroaimant et du haut-parleur \\
			S5 & Mesures fréquence de coupure et recherches documentaires. \\
			S6 & Soudure du premier circuit imprimé. Bobinage d'une bobine mobile et une bobine fixe \\
			S7 & PRÉ-JURY : rédaction du pré-rapport \\
			S8 & Prototypes de membrane et tests du circuit imprimé \\
			S9 & Finalisation de la membrane \\
			S10 & Réalisation du caisson, tests et modélisation mécanique de la membrane \\
			S11 & Réalisation du deuxième haut-parleur suivant les plans du premier \\
			S12 & Centralisation de tous les documents et dernières mesures en laboratoire. Rédaction du rapport \\
			S13 & Slides pour le jury final et préparation de la défense orale \\
			S14 & Répétition de la défense orale \\
		\hline
	\end{tabular}
	\caption{Planning du projet.}
\end{table}

\subsection{Critique} Nous pouvons dire que nous nous sommes assez bien tenus au planning durant toute la 
première partie du quadrimestre. Arrivés en semaine 7, nous étions tout à fait dans les temps, et déjà 
extrêmement bien avancés dans la rédaction du rapport, étant donné que nous avions décidé d'en écrire un 
pour le pré-jury. Malheureusement, des problèmes répétitifs en ce qui concerne le circuit imprimé nous ont
mis en retard. De tels évènements nous ont fait perdre énormément de temps, et c'est ainsi que notre 
haut-parleur n'a pas pu être finalisé.

% Just here to fix rapport_prejury.tex
\end{document}
