\documentclass{article}

\usepackage[utf8]{inputenc}
\usepackage[T1]{fontenc}      
\usepackage[francais]{babel}
\usepackage{graphicx}
\usepackage{circuitikz}
\usepackage[squaren, Gray]{SIunits}
\usepackage{sistyle}
\usepackage[autolanguage]{numprint}
\usepackage{pgfplots}
\pgfplotsset{compat=1.9}
\usepackage{amsmath,amssymb,array}
\usepackage[top=2.5cm,bottom=2.5cm,right=2.5cm,left=2.5cm]{geometry}
\usepackage{url} 
\usepackage{tabularx}
\DeclareMathOperator{\dist}{d}
\newenvironment{abstract-fr}
{
	\begin{center}
		\textbf{Résumé} \\[0.5cm]
	\end{center}
}
{}

\newenvironment{abstract-en}
{
	\begin{center}
		\textbf{Summary} \\[0.5cm]
	\end{center}
}
{}
% New command pour la modélisation mécanique, tri à effectuer
\newcommand\fv[1]{{\bf #1}} % free vector
\newcommand\fvd[1]{\dot{\bf #1}} % free vector derivated
\newcommand\fvdd[1]{\ddot{\bf #1}} % free vector derivated
\newcommand\fvr[1]{\mathring{\bf #1}} % free vector relatively derivated
\newcommand\fvrr[1]{\overset{\circ\circ}{\bf #1}} % free vector relatively derivated
\newcommand\uv[1]{{\bf\hat{ #1}}} % unit vector
\newcommand\ui{{\bf\hat{I}}} % unit vector I
\newcommand\uj{{\bf\hat{J}}} % unit vector J
\newcommand\uk{{\bf\hat{K}}} % unit vector K
\newcommand\wrt[2]{\ensuremath{\tensor*[_{ #1}]{ #2}{}}} % With Respect To
\newcommand\wtr[3]{\ensuremath{\tensor*[_{ #1}]{ #2}{^{ #3}}}} % With Two Respect
\newcommand\omegaf{{\bm \omega}}
\newcommand\omegafr{\mathring{\bm \omega}}
\newcommand\omegafd{\dot{\bm \omega}}
\newcommand\omegaft{\tilde{\bm \omega}}
\newcommand\omegaftr{\mathring{\tilde{\bm \omega}}}
\newcommand\omegat{\tilde{\omega}}
\newcommand\omegatd{\tilde{\dot{\omega}}}
\newcommand\ine{{\bf I}}
\newcommand\st{{\bf L}}
\newcommand\pst{{\bf M}}
\newcommand\lm{{\bf N}}
\newcommand\am{{\bf H}}
\newcommand\amd{\dot{\am}}
\newcommand\fo{{\bf F}}
\newcommand\po{\mathcal{P}}
\newcommand\xg{\ensuremath{\fv{R}}}
\newcommand\xgd{\ensuremath{\fvd{R}}}
\newcommand\xgdd{\ensuremath{\fvdd{R}}}
\newcommand\dvec[1]{\dot{\vec{ #1}}}
\newcommand\ddvec[1]{\ddot{\vec{ #1}}}
\newcommand\qp{\dot{q}}
\newcommand\dqp{\Delta \dot{q}}
\usepackage{url} 
\usepackage{hyperref}
\hypersetup{
    colorlinks,
    citecolor=black,
    filecolor=black,
    linkcolor=black,
    urlcolor=black
}

\begin{document}

% A revoir
\section{Planning}

Au début du quadrimestre, il nous a été demandé de concevoir, réaliser et quantifier un système de haut-parleur. Notre travail
a été répartie en trois partie : les séances tutorées, les laboratoires et le travail autonome. Dés le début de notre projet,
 plusieurs tâches nous ont été demandé : réaliser une recherche biblographique, concevoir un cahier des charges,
dimensionner notre haut-parleur, ect. 


\subsection{Avant le pré-jury}
Après 6 semaines de travail sur notre projet du deuxième quadrimestre, voici où en était l'état d'avancement 
de notre projet:

Nous avons décidé après un brain-strorming en groupe de centrer notre recherche bibliographique sur la distorsion harmonique 
et la boucle de contre-réaction présente dans notre circuit. Nous nous sommes documentés durant plusieurs semaines pour avoir
une recherche bibliographique la plus complète.

Une des première tâche importante de notre projet est de réaliser un cahier des charges. Nous avons posé les fonctions principales
et les contraintes de notre projet lors de la première semaine. Durant l'avancement de notre projet, nous l'avons complété
 afin d'améliorer au mieux les fonctions et contraintes ainsi que poser les dimensions de notre haut-parleur.

Dés la première semaine, nous nous sommes familiarisé avec les appareils en laboratoire pour ensuite travailler sur les
deux filtres passe-haut et passe-bas de notre circuit. Nous avons mesurer les tensions de sortie de ces blocs en fonction
de différentes capacités et résistances. Nous avons ensuite commencé à souder une partie des composantes sur nos plaquettes.

En parallèle, nous avons réalisé l'analyse mathématique et physique du filtre passe-bas ainsi que le dimensionnement de la bobine
pour l'électro-aimant. Une fois les dimensions définies, nous avons pu commencer à bobiner les différentes bobines.

Une esquisse de la membrane pour le haut-parleur a été proposé tout en sachant que quelques modifications restait à faire.

\subsection{Après le pré-jury}

Dés la semaine 9, nous nous sommes plus concentré sur la fabrication même du haut-parleur : nous avons commencé par la
membrane en papier et tissus que nous avons testé par la suite. Nous avons aussi réaliser les caisons en bois contenant notre
circuit et l'électro-aimant.

Nous nous étions fixé pour la semaine 11 d'avoir fini de souder les deux plaquettes pour les tests de validation. Malheureusement,
ayant eu quelques problème avec notre première plaquette, nous avons préféré rester concentrer sur celle-ci et ne pas terminer
la seconde. 

Après les tests de validations, nous avons remis en commun tout le travail effectué les semaines précédentes pour s'atteler
à la rédaction de notre rapport.
Dans les semaines à venir, nous prévoyons de préparer notre défense oral (préparation des slides, répartitions des temps 
de paroles, ect.).

Globalement nous avons su respecter en temps les différentes de notre projet en répartissant la tâche de travail de manière
équilibré sur les différentes semaines. Néanmoins, les dernières semaines ont été plus chargé avec la construction du haut parleur 
et de la rédaction de notre rapport.

\textbf{Ancienne version}

\subsection{Ce qui restait à faire}

Pour les semaines d'après pré-jury nous avions orgagnisé notre temps de la manière suivante:

\begin{enumerate}
	\item{S9}: Finaliser la conception de la membrane et la tester.
	\item{S10}: Réaliser le caisson dans lequel on mettra notre circuit ainsi que la membrane.
	\item{S11}: Souder entièrement la deuxième plaque pour faire notre deuxième haut-parleur (si le premier haut-parleur fonctionne).
	\item{S11}: Réaliser le deuxième haut-parleur en suivant les plans du premier.
	\item{S12}: Centraliser tous les travaux (bien mettre tout en ordre) et rédiger le rapport final.
	\item{S13}: Slides pour le jury-final
	\item{S14}: Préparation de la défense orale et achever les slides.
\end{enumerate}

Nous pouvons dire que nous nous sommes assez bien tenu au planning durant toute la 2ème partie du quadrimestre. 
Mis à part que le deuxième haut-parleur n'a pas été entièrement réalisé ( nous préférions nous concentrer d'abord sur 
le premier), les étapes ont été réalisées en temps et en heure.  La charge de travail était assez bien équilibrée sur 
les différentes semaines même si nous avons dû travailler plus dur dans les dernières semaines pour construire le haut-parleur
et rédiger le rapport.
% Just here to fix rapport_prejury.tex
\end{document}
