\documentclass{article}

\usepackage[utf8]{inputenc}
\usepackage[T1]{fontenc}      
\usepackage[francais]{babel}
\usepackage{graphicx}
\usepackage{circuitikz}
\usepackage[squaren, Gray]{SIunits}
\usepackage{sistyle}
\usepackage[autolanguage]{numprint}
\usepackage{pgfplots}
\pgfplotsset{compat=1.9}
\usepackage{amsmath,amssymb,array}
\usepackage[top=2.5cm,bottom=2.5cm,right=2.5cm,left=2.5cm]{geometry}
\usepackage{url} 
\usepackage{tabularx}
\DeclareMathOperator{\dist}{d}
\newenvironment{abstract-fr}
{
	\begin{center}
		\textbf{Résumé} \\[0.5cm]
	\end{center}
}
{}

\newenvironment{abstract-en}
{
	\begin{center}
		\textbf{Summary} \\[0.5cm]
	\end{center}
}
{}
% New command pour la modélisation mécanique, tri à effectuer
\newcommand\fv[1]{{\bf #1}} % free vector
\newcommand\fvd[1]{\dot{\bf #1}} % free vector derivated
\newcommand\fvdd[1]{\ddot{\bf #1}} % free vector derivated
\newcommand\fvr[1]{\mathring{\bf #1}} % free vector relatively derivated
\newcommand\fvrr[1]{\overset{\circ\circ}{\bf #1}} % free vector relatively derivated
\newcommand\uv[1]{{\bf\hat{ #1}}} % unit vector
\newcommand\ui{{\bf\hat{I}}} % unit vector I
\newcommand\uj{{\bf\hat{J}}} % unit vector J
\newcommand\uk{{\bf\hat{K}}} % unit vector K
\newcommand\wrt[2]{\ensuremath{\tensor*[_{ #1}]{ #2}{}}} % With Respect To
\newcommand\wtr[3]{\ensuremath{\tensor*[_{ #1}]{ #2}{^{ #3}}}} % With Two Respect
\newcommand\omegaf{{\bm \omega}}
\newcommand\omegafr{\mathring{\bm \omega}}
\newcommand\omegafd{\dot{\bm \omega}}
\newcommand\omegaft{\tilde{\bm \omega}}
\newcommand\omegaftr{\mathring{\tilde{\bm \omega}}}
\newcommand\omegat{\tilde{\omega}}
\newcommand\omegatd{\tilde{\dot{\omega}}}
\newcommand\ine{{\bf I}}
\newcommand\st{{\bf L}}
\newcommand\pst{{\bf M}}
\newcommand\lm{{\bf N}}
\newcommand\am{{\bf H}}
\newcommand\amd{\dot{\am}}
\newcommand\fo{{\bf F}}
\newcommand\po{\mathcal{P}}
\newcommand\xg{\ensuremath{\fv{R}}}
\newcommand\xgd{\ensuremath{\fvd{R}}}
\newcommand\xgdd{\ensuremath{\fvdd{R}}}
\newcommand\dvec[1]{\dot{\vec{ #1}}}
\newcommand\ddvec[1]{\ddot{\vec{ #1}}}
\newcommand\qp{\dot{q}}
\newcommand\dqp{\Delta \dot{q}}
\usepackage{url} 
\usepackage{hyperref}
\hypersetup{
    colorlinks,
    citecolor=black,
    filecolor=black,
    linkcolor=black,
    urlcolor=black
}

\begin{document}

\section{Planning}

\subsection{Ce qui est fait}
Après 6 semaines de travail sur notre projet Q2, jetons un oeil sur ce qui a déjà été éffectué et ce qui reste à faire jusqu'au jury final.

Nous pouvons dire que les points suivants ont été réalisés complètement :

\begin{enumerate}
	\item Nous avons fait les recherches documentaires sur la distorsion harmonique ainsi que sur la contre-réaction négative.
	Nous avons effectué une recherche bibliographique assez conséquente pour trouver de la documentation.
	\item Le cahier des charges est terminé, il comprend les fonctions, les contraintes et quelques dimensions de notre haut-parleur.
	\item Nous avons fait une analyse mathématique (fréquence limite, graphe) et physique (mesures en labo) du filtre passe-bas. 
	\item Nous avons fait une analyse mathématique (fréquence limite, graphe) et physique (mesures en labo) du filtre passe-haut.
	\item Nous avons mesuré le voltage sortant des filtres en fonction de la fréquence.
	\item Les dimensions de la bobine ont été calculées.
	\item Quelques parties du circuit ont été soudées.
	\item Une esquisse de la membrane du haut-parleur a été agencée, même si il reste quelques modifications à faire.
\end{enumerate}

\subsection{Ce qui restait à faire}

Pour les semaines d'après pré-jury nous avions orgagnisé notre temps de la manière suivante.

\begin{enumerate}
	\item{S9}: Finaliser la conception de la membrane et la tester.
	\item{S10}: Réaliser le caisson dans lequel on mettra notre circuit ainsi que la membrane.
	\item{S11}: Souder entièrement la deuxième plaque pour faire notre deuxième haut-parleur (si le premier haut-parleur fonctionne).
	\item{S11}: Réaliser le deuxième haut-parleur en suivant les plans du premier.
	\item{S12}: Centraliser tous les travaux (bien mettre tout en ordre) et rédiger le rapport final.
	\item{S13}: Slides pour le jury-final
	\item{S14}: Préparation de la défense orale et achever les slides.
\end{enumerate}

Nous pouvons dire que nous nous sommes assez bien tenu au planning durant toute la 2ème partie du quadrimestre. 
Mis à part que le deuxième haut-parleur n'a pas été entièrement réalisé ( nous préférions nous concentrer d'abord sur 
le premier), les étapes ont été réalisé en temps et en heure.  La charge de travail était assez bien équilibrée sur 
les différentes semaines même si nous avons dû travailler plus dur dans les dernières semaines pour construire le haut-parleur
et rédiger le rapport.
% Just here to fix rapport_prejury.tex
\end{document}
