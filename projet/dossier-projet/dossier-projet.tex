\documentclass{report}


\usepackage[latin1]{inputenc}
\usepackage[T1]{fontenc}
\usepackage[francais]{babel}
\usepackage{amsmath,amssymb,array}
 
      
\title{Dossier}
\author{Groupe 11.53}
\date{Juin 2014}
\begin{document}
 
\maketitle

Pour ce dossier projet, nous utiliserons la structure suivante:
\begin{enumerate}
\item Ce qui était prévu pour le laboratoire
\item Ce qui a été réalisé
\item Ce qui n'a pas été réalisé
\end{enumerate}

\section{Laboratoire 1 du 31 janvier 2014}
\begin{enumerate}
\item 
 Familiarisation avec les appareils présents en laboratoire(oscilloscope, générateur de fonction, multimètre,$\dots$)
\item 
 Nous n'avions jamais utilisé de tels appareils avant ce laboratoire. Une lecture des consignes et des datasheets nous a aidé à réalisé une bonne partie de ce qui était prévu. Cela comprend: mesure de tensions, de courants et de résistances à l'aide d'un multimètre, l'observation de circuits électriques à tension variable via un oscilloscope et un générateur de signaux. 
\item 
 Nous n'avons malheureusement pas eu le temps d'observer les circuits dynamiques correspondants aux filtres passe-haut et passe-bas (R-C et C-R)que cela soit sous tension continue autant que sous tension alternative.

\end{enumerate}

\section{Laboratoire 2 du 10 février 2014}
\begin{enumerate}
\item 
La priorité de ce laboratoire était de terminer ce qui avait été prévu mais non réalisé au précédant. Cela reprend l'observation des circuits dynamique (R-C et C-R), dans un premier temps sous tension continue, puis sous tension alternative. Nous voullions aussi essayer pour une première fois, sur une plaque d'essais, le fer à souder.

\item
 Nous avons réussi à réaliser tout ce que nous avions prévu, bien que nous ayons eu encore quelques difficultés avec l'utilisation de certains appareils comme l'oscilloscope. Après avoir mis la main à la pâte, l'utilisation du fer à souder nous parut moins compliquée que prévu, bien que nous manquions encore d'expérience pour réaliser de belles soudures.

\item 
Nous avons la chance de pouvoir laisser cette partie vide cette semaine.
\end{enumerate}

\section{Laboratoire 3 du 17 février 2014}

\begin{enumerate}

\item  Les PCB (Printed Circuit Board) sont arrivées. Nous souhaitions, dans un premier temps, tester le circuit complet du haut-parleur sur les breadboards.

\item Le résultat est là: nous arrivons, gràce à notre circuit et à une membrane se trouvant dans le
laboratoire, à produire un faible son.

\item Ce qui a été prévu a été réalisé.

\end{enumerate}

\section{Laboratoire 4 du 24 février 2014}
\begin{enumerate}
\item
 Nous allons attaquer la soudure des multiples composants(potentiomètres, ampli-op, capacités, jumpers, prise Jack). 

\item
Souhaitant ne pas nous tromper et abîmer notre matériel, nous avons, dans un premier temps, monter notre circuit sur la plaquette, sans souder ses composantes. C'est pourquoi nous n'allions guère vite à cette étape. Mais l'inévitable arriva. La jeunesse étant ce qu'elle est, des erreurs se sont produites. Erreurs 
que notre Tuteur Projet nous a fait remarquer. En effet, un potentiomètre était soudé à l'envers.    

\item Nous aurions voullu souder l'entièreté de la plaquette, malheureusement, dans le soucis de bien faire, nous avons travaillé dans une certaine lenteur. Le résultat fut que nous n'ayons soudé que la moitié de la plaquette.

\end{enumerate}

\section{Laboratoire 5 du 3 mars 2014}

\begin{enumerate}
\item
Lors de cette séance, nous avons deux objectifs: souder une plaquette et réfléchir à propos de notre membrane.
\item
Ayant déjà soudé, cette étape plus facile et plus rapide. Mais, malheureusement pour nous, une nouvelle erreur de jeunesse se produisit: quelques bouts metalliques des potentiomètres, capacités et autres, étaient connectés à la masse. Ce qui rend notre disquette inutilisable. \\ \\
La concepton de la membrane fut aussi plus compliquée que prévue: quelle géométrie dans le plan devions nous réaliser pour pouvoir plier la forme et obtenir la meilleur membrane possible.
\item Pour la partie "membrane", nous n'avons encore aucun résultat satisfaisant. 
 

\end{enumerate}

\section{Laboratoire 6 du 10 mars 2014}
\begin{enumerate}
\item Les objectifs restent les mêmes que ceux du laboratoire précédant.

\item Une plaquette a été soudée et des membranes provisoires en papier, de différentes tailles et de rigidité différentes ont été réalisées.

\item Ce que nous voullions réaliser l'a été. 

\end{enumerate}

\section{Laboratoire 7 du 17 mars 2014}
\begin{enumerate}

\item Nous objectif est d'analyser notre plaquette. Ainsi que réaliser la version finale de notre
membrane.
\item Malheureusement, nous n'avons pas le moindre son qui est produit, à partir de notre plaquette et de la membrane du labo. En ce qui concerne la memmbrane, elle est réalisée. 


\end{enumerate}

\section{Laboratoire 8 du 24 mars 2014}

\begin{enumerate}

\item
N'ayant pas eu un résultat positif la semaine passée, nous décidons de souder une nouvelle plaque, et de la testerlors du laboratoire.
\item Nous avons eu le temps de souder et de tester notre plaquette avec la membrane du labo. 
Eureka! Elle fonctionne, bien que le son soit assez faible.

\end{enumerate}
\section{Laboratoire 9 du 31 mars 2014}
\begin{enumerate}
\item Nous voullons, lors de cette séance, vérifier que notre plaquette fonctionne bien, que ce soit avec la membrane du labo, autant qu'avec notre caisson avec sa membrane en tissu. 
\item Malheureusement pour nous, le faible son obtenu au dernier labo ne se fait plus entendre
\end{enumerate}









Dossier : Séances tutorées.

\section{}
Séance1 : Enoncé du projet : Votre objectif est de réaliser, mesurer et qualifier un dispositif comportant un système d’amplification permettant d’écouter sur deux haut-parleurs de votre fabrication les signaux stéréo provenant de la fiche jack 3,5mm d’un smartphone ou d’un baladeur MP3 et d’en faire varier le volume, l’intensité des sons graves et aigus.
Nous avons réalisé un cahier des charges de notre haut-parleur, ainsi qu’un schéma fonctionnel de l’appareil.  Le cahier des charges comporte les contraintes, les caractéristiques et quelques dimensions de notre haut-parleur.  Le schéma fonctionnel était un plan de toutes les fonctions que devaient avoir pour que le haut-parleur fonctionne.  Cela nous a permis d’avoir une vision globale de la tâche qui nous attendait. Ne connaissant pas grand-chose sur le sujet, nous avons décidé de faire des recherches individuelles sur la fabrication d’un haut-parleur.  

\section{}
Séance 2 : Grâce à des APP en physique qui était surtout axés sur le projet, nous en savons un peu plus sur le fonctionnement d’un tel appareil.  Nous connaissons maintenant les composantes du système et pouvons évaluer les caractéristiques que doivent avoir chaque partie.  Ainsi, nous avons pu calculer le champ magnétique nécessaire pour faire vibrer la membrane.  Une partie du dimensionnement a donc été réalisé assez tôt dans ce projet.  Nous avons également du rechercher une série de mots-clefs en rapport avec le projet.

\section{}
Séance 3 : Nous avons analysé toutes les composantes du circuit et nous avons passé la plupart du temps à comprendre le fonctionnement de la plaquette.  Nous devions savoir comment les différentes parties étaient reliées entre elles.  Comment faire fonctionner la plaquette, avec quelle source de tension,…  Aussi, nous avons commencé  l’analyse mathématique des filtres passe-bas et passe-haut.  Cela n’a été possible qu’après avoir reçu le cours de math de Mr Vitale.  L’approximation mathématique nous a été d’une grande aide dans la résolution de ce problème.

\section{}
Séance 4 : Après avoir compris que nous avions besoins de données claires pour pouvoir avoir l’approximation, nous en avons réalisé au labo.  Avec ces données, nous avons pu les appliquer à la méthode que nous avons choisie pour pouvoir obtenir la fréquence de coupure.  N’obtenant pas la même réponse à chaque fois nous avons dû nous employé pour pouvoir obtenir une réponse correct.  La modélisation du graphe sur un logiciel a permis de prouver que notre estimation était tout à fait correcte.  L’équation différentielle a particulièrement longue et fastidieuse à chercher car le cheminement était long et une erreur est si vite arrivée.

\section{}
Séance 5 : La recherche documentaire devenait une de nos priorités pour le pré-jury qui arrivait.  Nous avons donc été à la bibliothèque par groupe de deux pour y prendre des livres relatant du sujet choisis.  Cela était une première pour la plupart des gens de notre groupe.  Aller sur le site libellule, trouver la référence d’un livre et le consulter de manière efficace pour en trouver l’information n’a pas été une mince affaire mais finalement nous nous trouvions instruit de nouveau termes.  Durant cette séance, nous avons mis en commun les recherches et nous avons finalisé les rapports des deux mots clefs.  

\section{}
Séance 6 : Dernière séance avant le pré-jury donc nous avons répartis le travail qu’il fallait faire pour que tout soit prêt pour le passage devant ce pré-jury.  Nous avons collecté toutes les informations que nous avions pour en faire un rapport amoindri.   Ce qui nous a permis de gagner du temps lors de la rédaction finale du rapport.  Nous avons également approfondis le dimensionnement de des bobines en leur donnant leur taille et caractéristiques finales.  

\section{}
Séance 7 : Pré-jury.  Nous avons défendu notre projet et nous avons vu où était nos erreurs et nos points forts.  

\section{}
Séance 8-9-10 : A partir de cette séance, les séances tutorées ont été moins calculatoire mais plus une préparation en profondeur de ce que nous devions faire en labo.  En effet, presque toute la théorie était réalisée, il fallait savoir pertinemment ce que nous allions faire à la prochaine séance.   Les séances servaient à poser des questions au tuteur et à comprendre ce qui se passait quand quelque chose ne marchait pas au labo.

\end{document}

