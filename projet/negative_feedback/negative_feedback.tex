\documentclass{article}

\usepackage[utf8]{inputenc}
\usepackage[T1]{fontenc}      
\usepackage[francais]{babel}
\usepackage{graphicx}
\usepackage{circuitikz}
\usepackage[squaren, Gray]{SIunits}
\usepackage{sistyle}
\usepackage[autolanguage]{numprint}
\usepackage{pgfplots}
\pgfplotsset{compat=1.9}
\usepackage{amsmath,amssymb,array}
\usepackage[top=2.5cm,bottom=2.5cm,right=2.5cm,left=2.5cm]{geometry}
\usepackage{url} 
\usepackage{tabularx}
\DeclareMathOperator{\dist}{d}
\newenvironment{abstract-fr}
{
	\begin{center}
		\textbf{Résumé} \\[0.5cm]
	\end{center}
}
{}

\newenvironment{abstract-en}
{
	\begin{center}
		\textbf{Summary} \\[0.5cm]
	\end{center}
}
{}
% New command pour la modélisation mécanique, tri à effectuer
\newcommand\fv[1]{{\bf #1}} % free vector
\newcommand\fvd[1]{\dot{\bf #1}} % free vector derivated
\newcommand\fvdd[1]{\ddot{\bf #1}} % free vector derivated
\newcommand\fvr[1]{\mathring{\bf #1}} % free vector relatively derivated
\newcommand\fvrr[1]{\overset{\circ\circ}{\bf #1}} % free vector relatively derivated
\newcommand\uv[1]{{\bf\hat{ #1}}} % unit vector
\newcommand\ui{{\bf\hat{I}}} % unit vector I
\newcommand\uj{{\bf\hat{J}}} % unit vector J
\newcommand\uk{{\bf\hat{K}}} % unit vector K
\newcommand\wrt[2]{\ensuremath{\tensor*[_{ #1}]{ #2}{}}} % With Respect To
\newcommand\wtr[3]{\ensuremath{\tensor*[_{ #1}]{ #2}{^{ #3}}}} % With Two Respect
\newcommand\omegaf{{\bm \omega}}
\newcommand\omegafr{\mathring{\bm \omega}}
\newcommand\omegafd{\dot{\bm \omega}}
\newcommand\omegaft{\tilde{\bm \omega}}
\newcommand\omegaftr{\mathring{\tilde{\bm \omega}}}
\newcommand\omegat{\tilde{\omega}}
\newcommand\omegatd{\tilde{\dot{\omega}}}
\newcommand\ine{{\bf I}}
\newcommand\st{{\bf L}}
\newcommand\pst{{\bf M}}
\newcommand\lm{{\bf N}}
\newcommand\am{{\bf H}}
\newcommand\amd{\dot{\am}}
\newcommand\fo{{\bf F}}
\newcommand\po{\mathcal{P}}
\newcommand\xg{\ensuremath{\fv{R}}}
\newcommand\xgd{\ensuremath{\fvd{R}}}
\newcommand\xgdd{\ensuremath{\fvdd{R}}}
\newcommand\dvec[1]{\dot{\vec{ #1}}}
\newcommand\ddvec[1]{\ddot{\vec{ #1}}}
\newcommand\qp{\dot{q}}
\newcommand\dqp{\Delta \dot{q}}
\usepackage{url} 
\usepackage{hyperref}
\hypersetup{
    colorlinks,
    citecolor=black,
    filecolor=black,
    linkcolor=black,
    urlcolor=black
}

\begin{document}

\section{La contre-réaction ou réaction négative}
En analysant le circuit de notre haut-parleur, nous avons découvert la présence de boucles reliant 
la sortie et la borne négative des amplificateurs. Nous nous sommes alors interrogés sur le rôle de ces boucles.

Nous allons dans un premier temps expliquer les raisons d'être des boucles de contre-réaction en général et 
nous finirons par l'explication complète de leur raison d'être dans le cas particulier de notre circuit.

\subsection{Principe de la réaction}
Le principe de la réaction est présent dans un grand nombre de circuits électroniques. Il consiste en une 
réinjection d'une partie du signal de sortie à l'entrée du circuit pour le combiner avec le signal d'entrée 
extérieur\cite{correvon}.

Il existe deux types de réactions\cite{correvon} :

\begin{itemize}
	\item \textbf{La réaction positive} : le signal réinjecté est en phase avec le signal d'entrée de telle 
	sorte que les deux signaux s'additionnent ;
	\item \textbf{La réaction négative} (ou contre-réaction) : le signal réinjecté est en opposition de 
	phase avec le signal d'entrée, de telle sorte que les deux signaux
	se soustraient.
\end{itemize}

\begin{figure}[h]
	\centering
	\begin{circuitikz}
		\draw (0, 0) node[ocirc];
		\draw (0, 0)	to[short] (2, 0);
		\draw (0, -1) node[ocirc];
		\draw (0, -1) to[short] (2, -1);
		\draw (3.1, -0.5) node [op amp, yscale=-1.022] (op amp) {}
					(opamp.-)node[left]
					(opamp.+)node[left]
					(opamp.out)node[right];
		\draw (3.85, -0.5) to[short] (5.6, -0.5);
		\draw (5.6, -0.5) node[ocirc];
		\draw (5.4, -0.5) to[short] (5.4, -2);
		\draw (5.4, -2) to[short] (1.4, -2);
		\draw (1.4, -2) to[short] (1.4, -1);
	\end{circuitikz}
	\caption{Schéma électrique d'une boucle de réaction sur un 	amplificateur.}
	\label{reaction1}
\end{figure}

\subsection{Effets des boucles de contre-réaction}

\subsubsection{En général}
Les effets des boucles de contre-réaction sur un amplificateur sont nombreux\cite{sporken}\cite{dusausay} :

\begin{itemize}
	\item La boucle de contre-réaction rend indépendant le gain de l'amplificateur des différentes variations du circuit ;
	\item Le signal de sortie est plus proche du signal d'entrée que si l'amplificateur avait été en boucle ouverte ;
	\item Réduction des signaux électriques parasites et de la distorsion dûs à l'amplificateur : en boucle ouverte, 
	le taux de distorsion d'un amplificateur est typiquement de 1\%. La boucle de contre-réaction permet de diminuer ce taux à 0.001\% ;
	\item Contrôle du gain de l'amplificateur (qui est, en boucle ouverte, de l'ordre de $10^6$) ;
	\item Élargissement de la bande passante de l'amplificateur ;
	\item Réduction de l'impédance de sortie.
\end{itemize}

\subsubsection{Intégration dans le circuit du haut-parleur}
Dans notre cas particulier, le principal effet de la boucle de contre-réaction est le contrôle du gain de l'amplificateur 
qui ramène à $1$ le gain.

\begin{figure}[h]
	\centering
	\begin{circuitikz}
		\draw (0,0) node[ocirc];
		\draw (3,0) to[short] (opamp+);
		\draw (4, -0.5) node [op amp, yscale=-1.022] (op amp) {}
			(opamp.-)node[left] (opamp-)
			(opamp.+)node[left] (opamp+)
			(opamp.out)node[right] (opampout);
		\draw (5, -0.5) to[short] (7, -0.5);
		\draw (7, -0.5) node[ocirc];
		\draw (2, -1) to[short] (3, -1);
		\draw (2, -1) to[short] (2, -3);
		\draw (2, -3) to[R=$R_1$] (2, -4);
		\draw (2, -4) to[short] (2, -4.5);
		\draw (2, -4) node[ground];
		\draw (2, -2) to[short] (6, -2);
		\draw (6, -2) to[R=$R_2$] (6, -0.5);
	\end{circuitikz}
	\caption{Schéma électrique d'une boucle de réaction sur un 	amplificateur avec un diviseur résistif.}
	\label{reaction2}
\end{figure}

Sur la Figure \ref{reaction2}, nous remarquons que la tension de sortie et la tension d'entrée sont liées 
par la formule des diviseurs résistifs :

$$V_{in} = \frac{R_1}{R_1 + R_2} V_{out}$$

Le gain est alors donné par :

$$A = \frac{V_{out}}{V_{in}} = \frac{R_1 + R_2}{R_1}$$

Pour réduire le gain $A$ à $1$, deux possibilités s'offrent à nous:

\begin{enumerate}
	\item	Choisir $R_1 >> R_2$ ;
	\item Choisir $R_2 = 0$ ;
\end{enumerate}

La possibilité la plus simple est la deuxième, car en choississant $R_2 = 0$, le gain est donné par $\frac{R_1}{R_1}$. 
Autrement dit : quelque soit $R_1$, on a $A = 1$ de telle sorte que $V_{in} = V_{out}$. On choisit alors $R_1$ si petit 
que le remplacer par un simple court-circuit a le même effet.

Dans un tel montage (appelé \textit{suiveur de tension}), la résistance d'entrée est infinie alors que la résistance de 
sortie est faible. Le courant de sortie est alors plus grand que le courant d'entrée (qui est presque nul).

Dans notre circuit, ces suiveurs de tension ont un rôle important puisqu'ils permettent le règlage indépendant des
graves et des aigus. Sans eux, modifier la résitance dans le filtre passe-bas modifierait aussi la résistance dans
le filtre passe-haut.

% Just here to fix rapport_prejury.tex
\end{document}
