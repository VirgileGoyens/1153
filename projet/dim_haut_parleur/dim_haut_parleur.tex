\documentclass{article}

\usepackage[utf8]{inputenc}
\usepackage[T1]{fontenc}      
\usepackage[francais]{babel}
\usepackage{graphicx}
\usepackage{circuitikz}
\usepackage[squaren, Gray]{SIunits}
\usepackage{sistyle}
\usepackage[autolanguage]{numprint}
\usepackage{pgfplots}
\pgfplotsset{compat=1.9}
\usepackage{amsmath,amssymb,array}
\usepackage[top=2.5cm,bottom=2.5cm,right=2.5cm,left=2.5cm]{geometry}
\usepackage{url} 
\usepackage{tabularx}
\DeclareMathOperator{\dist}{d}
\newenvironment{abstract-fr}
{
	\begin{center}
		\textbf{Résumé} \\[0.5cm]
	\end{center}
}
{}

\newenvironment{abstract-en}
{
	\begin{center}
		\textbf{Summary} \\[0.5cm]
	\end{center}
}
{}
% New command pour la modélisation mécanique, tri à effectuer
\newcommand\fv[1]{{\bf #1}} % free vector
\newcommand\fvd[1]{\dot{\bf #1}} % free vector derivated
\newcommand\fvdd[1]{\ddot{\bf #1}} % free vector derivated
\newcommand\fvr[1]{\mathring{\bf #1}} % free vector relatively derivated
\newcommand\fvrr[1]{\overset{\circ\circ}{\bf #1}} % free vector relatively derivated
\newcommand\uv[1]{{\bf\hat{ #1}}} % unit vector
\newcommand\ui{{\bf\hat{I}}} % unit vector I
\newcommand\uj{{\bf\hat{J}}} % unit vector J
\newcommand\uk{{\bf\hat{K}}} % unit vector K
\newcommand\wrt[2]{\ensuremath{\tensor*[_{ #1}]{ #2}{}}} % With Respect To
\newcommand\wtr[3]{\ensuremath{\tensor*[_{ #1}]{ #2}{^{ #3}}}} % With Two Respect
\newcommand\omegaf{{\bm \omega}}
\newcommand\omegafr{\mathring{\bm \omega}}
\newcommand\omegafd{\dot{\bm \omega}}
\newcommand\omegaft{\tilde{\bm \omega}}
\newcommand\omegaftr{\mathring{\tilde{\bm \omega}}}
\newcommand\omegat{\tilde{\omega}}
\newcommand\omegatd{\tilde{\dot{\omega}}}
\newcommand\ine{{\bf I}}
\newcommand\st{{\bf L}}
\newcommand\pst{{\bf M}}
\newcommand\lm{{\bf N}}
\newcommand\am{{\bf H}}
\newcommand\amd{\dot{\am}}
\newcommand\fo{{\bf F}}
\newcommand\po{\mathcal{P}}
\newcommand\xg{\ensuremath{\fv{R}}}
\newcommand\xgd{\ensuremath{\fvd{R}}}
\newcommand\xgdd{\ensuremath{\fvdd{R}}}
\newcommand\dvec[1]{\dot{\vec{ #1}}}
\newcommand\ddvec[1]{\ddot{\vec{ #1}}}
\newcommand\qp{\dot{q}}
\newcommand\dqp{\Delta \dot{q}}
\usepackage{url} 
\usepackage{hyperref}
\hypersetup{
    colorlinks,
    citecolor=black,
    filecolor=black,
    linkcolor=black,
    urlcolor=black
}

\begin{document}

\begin{document}

\section{Dimensionnement du haut-parleur}

Après avoir réalisé quelques recherches sur les haut-parleurs, nous avons pu imaginer le dispositif idéal
à réaliser. En tenant compte des différentes contraintes qui nous étaient imposées, voici les différents
choix que nous avons effectués.

\subsection{Le boîtier}
La première question qui s'est posée était celle du volume du caisson. Or le volume du caisson ($V_b$) est
lié à la fréquence de résonance du
haut-parleur à l'air libre ($F_s$) , la fréquence de résonance du haut-parleur fermé ($F_b$), et au volume
d'air équivalent à la suspension du haut-parleur\footnote{"Représente le volume auquel serait comprimé
\unit{1}{\meter\cubed} d'air pour exercer une force équivalente à la compliance (inverse de la raideur) de la suspension"\cite{Vas}.}
($V_{as}$) , selon l'équation suivante\cite{Vas}:

$$\frac{F_b}{F_s} = \sqrt{\frac{V_{as}}{V_b} + 1}$$

Nous remarquons qu'un haut-parleur idéal serait de volume infini, étant donné que si $V_b\rightarrow \infty$, la fréquence de résonance serait égale à celle à l'air libre. Mais nous ne voulions pas d'un caisson trop grand, pour des questions pratiques et esthétiques.
Le fait que la fréquence de résonance d'un haut-parleur fermé soit plus élevée que celle à l'air libre implique que
la fréquence de coupure du passe-haut est également plus élevée. Cela a pour conséquence
qu'un volume de caisson trop petit ne restitue pas les extrêmes graves. Nous avons finalement opté pour un boîtier
cubique de $\unit{25}{\centi\meter}$ de côté.

Afin d'améliorer un peu le boîtier, nous avons également pensé à placer des pieds en caoutchouc afin de
réduire les déplacements dûs aux vibrations du haut-parleur. Nous avions également pensé placer un évent à l'avant du haut-parleur
pour augmenter le rendement en profitant de l'onde arrière, mais c'était plus difficile à construire, et
il aurait fallu que l'on accorde l'event, de manière à exploiter l'onde arrière correctement. Nous nous
sommes donc finalement limités à une charge\footnote{Manière de séparer les ondes avant et arrière.} dite
\textit{close}\cite{close}.

Enfin, pour faciliter l'accès à l'intérieur du haut-parleur, nous avons créé un système de porte coulissante
à l'arrière de celui-ci.

\subsection{La membrane}
Nous avons opté pour une membrane de diamètre de $\unit{17}{\centi\meter}$. Nous avons choisi cette valeur afin
d'avoir une membrane assez large, pour exploiter le mieux possible la taille du caisson. C'est également un
diamètre assez répandu dans le commerce\cite{tlhp}. Nous respectons donc les normes.
La profondeur de la membrane est de $\unit{6}{\centi\meter}$, comme pour la plupart des membranes de ce
diamètre\cite{tlhp}. Elle est réalisée en papier, et nous avons opté pour du tissus tendu en guise de ressort.
Cette solution nous apparaît comme sortant de l'ordinaire, propre, et efficace. Cela nous a en effet permis
d'obtenir une constante de raideur minime.

\begin{table}[htb!]
	\centering
	\begin{tabularx}{\textwidth}{|X|X|}
	\hline
	\textbf{Caractéristique} & \textbf{Justification} \\
	\hline
	Volume du caisson : $\unit{25\times25\times25}{\centi\meter}$ & Possibilité de faire varier les fréquences. \\
	\hline
	Matériau du caisson : Panneau de MDF
	d'épaisseur \unit{18}{\milli\meter} & Qualité, robustesse et coût. \\
	\hline
	Diamètre de membrane : \unit{17}{\centi\meter} & Avoir une membrane assez large pour exploiter le mieux possible la taille du caisson. \\
	\hline
	Profondeur de la membrane : \unit{6}{\centi\meter} & Déterminé en fonction du diamètre de la membrane. \\
	\hline
	Materiau membrane : papier et tissus & Rigidité et petite constante de raideur. \\
	\hline
	Masse surfacique du papier : \unit{200}{\gram\per\meter\squared} & Rigidité et coût. \\
	\hline
	\end{tabularx}
	\caption{Tableau récapitulatif.}
\end{table}
% Just here to fix rapport_prejury.tex
\end{document}
