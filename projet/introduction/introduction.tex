\documentclass{article}

\usepackage[utf8]{inputenc}
\usepackage[T1]{fontenc}      
\usepackage[francais]{babel}
\usepackage{graphicx}
\usepackage{circuitikz}
\usepackage[squaren, Gray]{SIunits}
\usepackage{sistyle}
\usepackage[autolanguage]{numprint}
\usepackage{pgfplots}
\pgfplotsset{compat=1.9}
\usepackage{amsmath,amssymb,array}
\usepackage[top=2.5cm,bottom=2.5cm,right=2.5cm,left=2.5cm]{geometry}
\usepackage{url} 
\usepackage{tabularx}
\DeclareMathOperator{\dist}{d}
\newenvironment{abstract-fr}
{
	\begin{center}
		\textbf{Résumé} \\[0.5cm]
	\end{center}
}
{}

\newenvironment{abstract-en}
{
	\begin{center}
		\textbf{Summary} \\[0.5cm]
	\end{center}
}
{}
% New command pour la modélisation mécanique, tri à effectuer
\newcommand\fv[1]{{\bf #1}} % free vector
\newcommand\fvd[1]{\dot{\bf #1}} % free vector derivated
\newcommand\fvdd[1]{\ddot{\bf #1}} % free vector derivated
\newcommand\fvr[1]{\mathring{\bf #1}} % free vector relatively derivated
\newcommand\fvrr[1]{\overset{\circ\circ}{\bf #1}} % free vector relatively derivated
\newcommand\uv[1]{{\bf\hat{ #1}}} % unit vector
\newcommand\ui{{\bf\hat{I}}} % unit vector I
\newcommand\uj{{\bf\hat{J}}} % unit vector J
\newcommand\uk{{\bf\hat{K}}} % unit vector K
\newcommand\wrt[2]{\ensuremath{\tensor*[_{ #1}]{ #2}{}}} % With Respect To
\newcommand\wtr[3]{\ensuremath{\tensor*[_{ #1}]{ #2}{^{ #3}}}} % With Two Respect
\newcommand\omegaf{{\bm \omega}}
\newcommand\omegafr{\mathring{\bm \omega}}
\newcommand\omegafd{\dot{\bm \omega}}
\newcommand\omegaft{\tilde{\bm \omega}}
\newcommand\omegaftr{\mathring{\tilde{\bm \omega}}}
\newcommand\omegat{\tilde{\omega}}
\newcommand\omegatd{\tilde{\dot{\omega}}}
\newcommand\ine{{\bf I}}
\newcommand\st{{\bf L}}
\newcommand\pst{{\bf M}}
\newcommand\lm{{\bf N}}
\newcommand\am{{\bf H}}
\newcommand\amd{\dot{\am}}
\newcommand\fo{{\bf F}}
\newcommand\po{\mathcal{P}}
\newcommand\xg{\ensuremath{\fv{R}}}
\newcommand\xgd{\ensuremath{\fvd{R}}}
\newcommand\xgdd{\ensuremath{\fvdd{R}}}
\newcommand\dvec[1]{\dot{\vec{ #1}}}
\newcommand\ddvec[1]{\ddot{\vec{ #1}}}
\newcommand\qp{\dot{q}}
\newcommand\dqp{\Delta \dot{q}}
\usepackage{url} 
\usepackage{hyperref}
\hypersetup{
    colorlinks,
    citecolor=black,
    filecolor=black,
    linkcolor=black,
    urlcolor=black
}

\begin{document}

% L'objectif du projet + les spécifications attendues + cahier des charges (ref)
Dans le cadre du cours \textit{Projet 2} du deuxième quadrimestre, notre groupe a été amené à 
concevoir un haut-parleur connectable, via une prise Jack \unit{3.5}{\milli\meter}, à un 
GSM ou un MP3 (les contraintes et spécifications sont détaillées dans l'annexe ''Cahier des charges''). 
En plus de cela, notre haut-parleur doit permettre un règlage du volume, des graves
et des aigus. Un autre objectif du projet est d'apprendre à travailler et à s'organiser \textit{en groupe}, 
comme le font tous les jours les ingénieurs.

% L'organisation du rapport
Ce rapport s'articule principalement en deux grands chapitres. Le premier
rassemble les différentes étapes de modélisations mathématiques
et physiques de composants du haut-parleur. Dans ce chapitre, nous 
commencerons par une vue générale du haut-parleur qui nous
permettra d'introduire les concepts physiques clés. Nous continuerons
ensuite par la modélisation des filtres passe-bas, passe-haut et passe-bande.
Après cela, nous nous attarderons  sur le dimensionnement de l'électroaimant et de la bobine mobile pour enfin terminer par la modélisation mécanique de la bobine
mobile et le dimensionnement du haut-parleur.

Le deuxième chapitre contient quant à lui la synthèse des recherches documentaires
effectuées. Ces recherches portent sur deux sujets liés à notre haut-parleur. Le premier
concerne plutôt l'acoustique; il s'agit de la 
distorsion harmonique. Le deuxième quant à lui concerne un concept lié au circuit électrique qui
compose notre haut-parleur; il s'agit du principe de la contre-réaction.

% Description générale du système
Penchons-nous sans plus tarder sur une description générale du système\footnote{Une
description plus détaillée sera faite dans le chapitre suivant.} :

Le GSM ou le MP3 connecté au haut-parleur via le cable Jack va, dans un premier temps, envoyer un signal audio dans le 
circuit imprimé. Ce signal peut être modifié de trois façons :

\begin{itemize}
	\item En règlant le volume, c'est-à-dire en modifiant l'amplitude du signal audio ;
	\item En règlant les graves et les aigus, c'est-à-dire en atténuant les basses ou les hautes
	fréquences. Il s'agit du rôle des filtres passe-bas et passe-haut qui, combinés, forment un filtre 
	passe-bande ;
 \item En amplifiant le signal : c'est le rôle de l'amplificateur audio du circuit.
\end{itemize}
 
À la sortie du circuit imprimé, le signal \textit{filtré} et \textit{amplifié} va alimenter en courant la bobine mobile.
Cette dernière va intercepter le champ magnétique constant produit par l'électroaimant, dimensionné préalablement
pour répondre à notre cahier des charges. La bobine mobile va subir une force qui déplacera la membrane en fonction
du signal audio et produira le son voulu.

% Just here to fix rapport_prejury.tex
\end{document}
