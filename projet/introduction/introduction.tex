\documentclass{article}

\usepackage[utf8]{inputenc}
\usepackage[T1]{fontenc}      
\usepackage[francais]{babel}
\usepackage{graphicx}
\usepackage{circuitikz}
\usepackage[squaren, Gray]{SIunits}
\usepackage{sistyle}
\usepackage[autolanguage]{numprint}
\usepackage{pgfplots}
\pgfplotsset{compat=1.9}
\usepackage{amsmath,amssymb,array}
\usepackage[top=2.5cm,bottom=2.5cm,right=2.5cm,left=2.5cm]{geometry}
\usepackage{url} 
\usepackage{tabularx}
\DeclareMathOperator{\dist}{d}
\newenvironment{abstract-fr}
{
	\begin{center}
		\textbf{Résumé} \\[0.5cm]
	\end{center}
}
{}

\newenvironment{abstract-en}
{
	\begin{center}
		\textbf{Summary} \\[0.5cm]
	\end{center}
}
{}
% New command pour la modélisation mécanique, tri à effectuer
\newcommand\fv[1]{{\bf #1}} % free vector
\newcommand\fvd[1]{\dot{\bf #1}} % free vector derivated
\newcommand\fvdd[1]{\ddot{\bf #1}} % free vector derivated
\newcommand\fvr[1]{\mathring{\bf #1}} % free vector relatively derivated
\newcommand\fvrr[1]{\overset{\circ\circ}{\bf #1}} % free vector relatively derivated
\newcommand\uv[1]{{\bf\hat{ #1}}} % unit vector
\newcommand\ui{{\bf\hat{I}}} % unit vector I
\newcommand\uj{{\bf\hat{J}}} % unit vector J
\newcommand\uk{{\bf\hat{K}}} % unit vector K
\newcommand\wrt[2]{\ensuremath{\tensor*[_{ #1}]{ #2}{}}} % With Respect To
\newcommand\wtr[3]{\ensuremath{\tensor*[_{ #1}]{ #2}{^{ #3}}}} % With Two Respect
\newcommand\omegaf{{\bm \omega}}
\newcommand\omegafr{\mathring{\bm \omega}}
\newcommand\omegafd{\dot{\bm \omega}}
\newcommand\omegaft{\tilde{\bm \omega}}
\newcommand\omegaftr{\mathring{\tilde{\bm \omega}}}
\newcommand\omegat{\tilde{\omega}}
\newcommand\omegatd{\tilde{\dot{\omega}}}
\newcommand\ine{{\bf I}}
\newcommand\st{{\bf L}}
\newcommand\pst{{\bf M}}
\newcommand\lm{{\bf N}}
\newcommand\am{{\bf H}}
\newcommand\amd{\dot{\am}}
\newcommand\fo{{\bf F}}
\newcommand\po{\mathcal{P}}
\newcommand\xg{\ensuremath{\fv{R}}}
\newcommand\xgd{\ensuremath{\fvd{R}}}
\newcommand\xgdd{\ensuremath{\fvdd{R}}}
\newcommand\dvec[1]{\dot{\vec{ #1}}}
\newcommand\ddvec[1]{\ddot{\vec{ #1}}}
\newcommand\qp{\dot{q}}
\newcommand\dqp{\Delta \dot{q}}
\usepackage{url} 
\usepackage{hyperref}
\hypersetup{
    colorlinks,
    citecolor=black,
    filecolor=black,
    linkcolor=black,
    urlcolor=black
}

\begin{document}

% L'objectif du projet + les spécifications attendues + cahier des charges (ref)
Dans le cadre du cours \textit{Projet 2} du deuxième quadrimestre, notre groupe a été amené à 
concevoir un haut-parleur connectable, via une prise Jack \unit{3.5}{\milli\meter}, à un 
GSM ou un MP3 (les contraintes et spécifications sont détaillées dans l'annexe ''Cahier des charges''). 
En plus de cela, notre haut-parleur doit permette un règlage du volume, des graves
et des aigus. Un autre objectif du projet est d'apprendre un travailler et à s'organiser \textit{en groupe}, 
comme le font tous les jours les ingénieurs.

% L'organisation du rapport
Ce rapport s'articule principalement en deux grands chapitres. Le premier
rassemble les différentes étapes de modélisations mathématiques
et physiques de composants du haut-parleur. Dans ce chapitre, nous 
commencerons par une vue générale du haut-parleur qui nous
permettra d'introduire les concepts physiques clés. Nous continuerons
ensuite par la modélisation des filtres passe-bas, passe-haut et passe bande.
Après cela, nous nous attarderons  sur le dimensionnement de l'électroaimant et
de la bobine mobile pour enfin terminer par la modélisation mécanique de la bobine
mobile.

Le deuxième chapitre contient quant à lui la synthèse des recherches documentaires
effectuées. Ces recherches portent sur deux sujets liés à notre haut-parleur. Le premier
concerne plutôt l'acoustique, il s'agit de la 
distorsion harmonique. Le deuxième quant à lui concerne un concept lié au circuit électrique qui
compose notre haut-parleur, il s'agit du principe de la contre-réaction.

% Description générale du système
Un haut-parleur est un outil permettant de transformer un signal 
électrique en un son. Grâce à un électro-aimant constituée d'une 
bobine fixe dans laquelle passe du courant, une bobine mobile 
oscille et fait vibrer la membrane à laquelle elle est attachée. 
Cette oscillation, qui dépend de la fréquence du signal électrique, 
crée une onde sonore.

Dans notre haut-parleur nous pouvons retrouver la membrane, la plaquette électrique, l
a bobine fixe et mobile, les sources de tension et le caisson.  Chaque partie a 
une particularité et ensemble, le haut-parleur peut remplir sa fonction.  Deux sources
de tension sont relié à la plaquette qui est elle-même relié au GSM d’une part et à 
la bobine mobile d’autre part.  Une autre source de tension permet à la bobine fixe de 
créer un champ magnétique constant.  La fréquence émise par le GSM permet à la bobine 
mobile de bouger selon la musique et la membrane vibre pour émettre le son souhaité.

% Just here to fix rapport_prejury.tex
\end{document}