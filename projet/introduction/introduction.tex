\documentclass{article}

\usepackage[utf8]{inputenc}
\usepackage[T1]{fontenc}      
\usepackage[francais]{babel}
\usepackage{graphicx}
\usepackage{circuitikz}
\usepackage[squaren, Gray]{SIunits}
\usepackage{sistyle}
\usepackage[autolanguage]{numprint}
\usepackage{pgfplots}
\pgfplotsset{compat=1.9}
\usepackage{amsmath,amssymb,array}
\usepackage[top=2.5cm,bottom=2.5cm,right=2.5cm,left=2.5cm]{geometry}
\usepackage{url} 
\usepackage{tabularx}
\DeclareMathOperator{\dist}{d}
\newenvironment{abstract-fr}
{
	\begin{center}
		\textbf{Résumé} \\[0.5cm]
	\end{center}
}
{}

\newenvironment{abstract-en}
{
	\begin{center}
		\textbf{Summary} \\[0.5cm]
	\end{center}
}
{}
% New command pour la modélisation mécanique, tri à effectuer
\newcommand\fv[1]{{\bf #1}} % free vector
\newcommand\fvd[1]{\dot{\bf #1}} % free vector derivated
\newcommand\fvdd[1]{\ddot{\bf #1}} % free vector derivated
\newcommand\fvr[1]{\mathring{\bf #1}} % free vector relatively derivated
\newcommand\fvrr[1]{\overset{\circ\circ}{\bf #1}} % free vector relatively derivated
\newcommand\uv[1]{{\bf\hat{ #1}}} % unit vector
\newcommand\ui{{\bf\hat{I}}} % unit vector I
\newcommand\uj{{\bf\hat{J}}} % unit vector J
\newcommand\uk{{\bf\hat{K}}} % unit vector K
\newcommand\wrt[2]{\ensuremath{\tensor*[_{ #1}]{ #2}{}}} % With Respect To
\newcommand\wtr[3]{\ensuremath{\tensor*[_{ #1}]{ #2}{^{ #3}}}} % With Two Respect
\newcommand\omegaf{{\bm \omega}}
\newcommand\omegafr{\mathring{\bm \omega}}
\newcommand\omegafd{\dot{\bm \omega}}
\newcommand\omegaft{\tilde{\bm \omega}}
\newcommand\omegaftr{\mathring{\tilde{\bm \omega}}}
\newcommand\omegat{\tilde{\omega}}
\newcommand\omegatd{\tilde{\dot{\omega}}}
\newcommand\ine{{\bf I}}
\newcommand\st{{\bf L}}
\newcommand\pst{{\bf M}}
\newcommand\lm{{\bf N}}
\newcommand\am{{\bf H}}
\newcommand\amd{\dot{\am}}
\newcommand\fo{{\bf F}}
\newcommand\po{\mathcal{P}}
\newcommand\xg{\ensuremath{\fv{R}}}
\newcommand\xgd{\ensuremath{\fvd{R}}}
\newcommand\xgdd{\ensuremath{\fvdd{R}}}
\newcommand\dvec[1]{\dot{\vec{ #1}}}
\newcommand\ddvec[1]{\ddot{\vec{ #1}}}
\newcommand\qp{\dot{q}}
\newcommand\dqp{\Delta \dot{q}}
\usepackage{url} 
\usepackage{hyperref}
\hypersetup{
    colorlinks,
    citecolor=black,
    filecolor=black,
    linkcolor=black,
    urlcolor=black
}

\begin{document}

\begin{document}

Dans le cadre du projet de bac 2, notre groupe a été amené à construire un haut-parleur fonctionnel permettant d’entendre un son sortant d’un Gsm ou mp3.  Durant tout le quadrimestre nous avons avancé par petites étapes.  Nous avons dû faire preuve d’organisation et de détermination pour achever ce projet.  Pour la plupart d’entre nous, cela était notre 2ème projet de groupe donc nous avons pu nous améliorer par rapport au premier projet même si il nous faut encore progresser. 

Le but précis de notre projet était de créer un haut-parleur qui puisse émettre le son sortant d’un Gsm ou mp3.  L’appareil doit aussi être capable de modifier le volume de ce son et de changer les aigus et les graves.  L’autre objectif recherché était de réaliser un projet en groupe, comme sont amené à le faire les ingénieurs.  

Dans ce rapport, on peut trouver les idées générales qui ont permis l’élaboration du haut-parleur.  Tel que la modélisation de l’appareil, les calculs mathématiques qui ont permis de définir cette modélisation, les recherches documentaires concernant les points plus subtils que nous avons voulu abordé, les différents blocs formant le haut-parleur en lui-même, la validation du système comprenant les tests effectués et les mesures prises […]  En annexe, tous les calculs nécessaires sont présents.

Dans notre haut-parleur nous pouvons retrouver la membrane, la plaquette électrique, la bobine fixe et mobile, les sources de tension et le caisson.  Chaque partie a une particularité et ensemble, le haut-parleur peut remplir sa fonction.  Deux sources de tension sont relié à la plaquette qui est elle-même relié au Gsm d’une part et à la bobine mobile d’autre part.  Une autre source de tension permet à la bobine fixe de créer un champ magnétique constant.  La fréquence émise par le Gsm permet à la bobine mobile de bouger selon la musique et la membrane vibre pour émettre le son souhaité.

\end{document}
% Just here to fix rapport_prejury.tex
\end{document}
