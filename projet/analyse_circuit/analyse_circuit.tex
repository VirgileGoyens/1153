\documentclass{article}

\usepackage[utf8]{inputenc}
\usepackage[T1]{fontenc}      
\usepackage[francais]{babel}
\usepackage{graphicx}
\usepackage{circuitikz}
\usepackage[squaren, Gray]{SIunits}
\usepackage{sistyle}
\usepackage[autolanguage]{numprint}
\usepackage{pgfplots}
\pgfplotsset{compat=1.9}
\usepackage{amsmath,amssymb,array}
\usepackage[top=2.5cm,bottom=2.5cm,right=2.5cm,left=2.5cm]{geometry}
\usepackage{url} 
\usepackage{tabularx}
\DeclareMathOperator{\dist}{d}
\newenvironment{abstract-fr}
{
	\begin{center}
		\textbf{Résumé} \\[0.5cm]
	\end{center}
}
{}

\newenvironment{abstract-en}
{
	\begin{center}
		\textbf{Summary} \\[0.5cm]
	\end{center}
}
{}
% New command pour la modélisation mécanique, tri à effectuer
\newcommand\fv[1]{{\bf #1}} % free vector
\newcommand\fvd[1]{\dot{\bf #1}} % free vector derivated
\newcommand\fvdd[1]{\ddot{\bf #1}} % free vector derivated
\newcommand\fvr[1]{\mathring{\bf #1}} % free vector relatively derivated
\newcommand\fvrr[1]{\overset{\circ\circ}{\bf #1}} % free vector relatively derivated
\newcommand\uv[1]{{\bf\hat{ #1}}} % unit vector
\newcommand\ui{{\bf\hat{I}}} % unit vector I
\newcommand\uj{{\bf\hat{J}}} % unit vector J
\newcommand\uk{{\bf\hat{K}}} % unit vector K
\newcommand\wrt[2]{\ensuremath{\tensor*[_{ #1}]{ #2}{}}} % With Respect To
\newcommand\wtr[3]{\ensuremath{\tensor*[_{ #1}]{ #2}{^{ #3}}}} % With Two Respect
\newcommand\omegaf{{\bm \omega}}
\newcommand\omegafr{\mathring{\bm \omega}}
\newcommand\omegafd{\dot{\bm \omega}}
\newcommand\omegaft{\tilde{\bm \omega}}
\newcommand\omegaftr{\mathring{\tilde{\bm \omega}}}
\newcommand\omegat{\tilde{\omega}}
\newcommand\omegatd{\tilde{\dot{\omega}}}
\newcommand\ine{{\bf I}}
\newcommand\st{{\bf L}}
\newcommand\pst{{\bf M}}
\newcommand\lm{{\bf N}}
\newcommand\am{{\bf H}}
\newcommand\amd{\dot{\am}}
\newcommand\fo{{\bf F}}
\newcommand\po{\mathcal{P}}
\newcommand\xg{\ensuremath{\fv{R}}}
\newcommand\xgd{\ensuremath{\fvd{R}}}
\newcommand\xgdd{\ensuremath{\fvdd{R}}}
\newcommand\dvec[1]{\dot{\vec{ #1}}}
\newcommand\ddvec[1]{\ddot{\vec{ #1}}}
\newcommand\qp{\dot{q}}
\newcommand\dqp{\Delta \dot{q}}
\usepackage{url} 
\usepackage{hyperref}
\hypersetup{
    colorlinks,
    citecolor=black,
    filecolor=black,
    linkcolor=black,
    urlcolor=black
}

\begin{document}

% Circuit électriquer à refaire en LaTeX, là ils sont vraiment
% dégeulasses.

\section{Analyse séquentielle du circuit}
Dans cette section, nous allons décrire le fonctionnement du circuit
de notre haut-parleur de la manière la plus précise et la plus complète
possible. 

Cette section est découpée en quatres sections, une pour chaque bloc 
principal du circuit. Chaque bloc est associé à un rôle bien précis du
haut-parleur.

\subsection{Bloc 1 : la connexion avec la prise Jack}

\begin{figure}[h]
	\centering
	\includegraphics[scale=0.6]{img/bloc1.png}
	\caption{Premier bloc du circuit, contenant 
					la prise jack femelle.}
	\label{bloc1}
\end{figure}

Sur ce premier bloc (Figure \ref{bloc1}), chargé de faire la connexion entre la source (smartphone, iPod, etc) et le reste du circuit, on a la prise Jack femelle. Pour bien comprendre son fonctionnement, regardons d'abord à quoi ressemble la prise Jack mâle dont nous disposons (Figure \ref{jack}).

\begin{figure}[h]
	\centering
	\includegraphics[scale=0.1]{img/jack.png}
	\caption{Prise Jack mâle 3.5mm, contact en 3 points.}
	\label{jack}
\end{figure}

Cette prise Jack mâke sert à transporter un signal stéréophonique, qui sépare le canal gauche et le canal droit.
Le canal gauche correspond à la pointe de la prise (L sur la figure), le canal droit correspond à l'anneau de la prise (R sur la figure). Le manchon de la prise correspond quant à lui à la masse.

Sur le schéma du bloc 1 (Figure \ref{bloc1}), on peut alors voir que le signal du canal droit sera traité par le circuit, tandis que le signal du canal gauche sera redirigé vers le point \textit{IN2} où il pourra être connecté à notre deuxième haut-parleur pour être traité à son tour.

\subsection{Bloc 2 : le réglage du volume}

Le fonctionnement de ce bloc est relativement simple à comprendre, il est constitué d'un potentiomètre (P1 sur la Figure \ref{bloc2}) dont la résistance varie entre $\unit{10}{\ohm}$ et $\unit{2 \cdot 10^{6}}{\ohm}$. Selon la valeur de la résistance, par la loi d'Ohm, l'amplitude du signal sera plus ou moins réduite et donc le volume sera plus ou moins grand.

\begin{figure}[h]
	\centering
	\includegraphics[scale=0.6]{img/bloc2.png}
	\caption{Schéma du bloc 2, chargé du réglage du volume.}
	\label{bloc2}
\end{figure}

\subsection{Bloc 3 : le réglage des graves et des aigus}

Ce bloc-ci est sans aucun doute le plus compliqué à comprendre. Il est constitué d'un filtre passe-haut (réglage des aigus) et d'un filtre passe-bas (réglage des graves). Le filtre passe-haut est celui qui suit le premier amplificateur (\textit{LM358N-A}), le passe-bas est celui qui suit le deuxième amplificateur (\textit{LM358N-B}). Ces deux amplificateurs sont ce qu'on appelle des amplificateurs \textit{suiveurs}. Leur rôle est de permettre le règlage des graves et des aigus de manière indépendantes. Sans ces amplificateurs suiveurs, faire varier le potentiomètre \textit{P2} influencerait non-seulement le filtre passe-haut, mais influencerait aussi le filtre passe-bas (car les potentiomètres \textit{P2} et \textit{P3} sont en série).

Un autre point intéressant à relever sur le schéma du bloc 3 est la présence d'une boucle reliant la sortie à la borne négative de chaque amplificateur. Ces boucles sont appelées "boucles de contre-réaction". Le gain normal d'un amplificateur est de l'ordre de $10^{6}$. Grâce aux boucles de contre-réaction, on peut contrôller le gain d'un amplificateur. Dans notre cas, le gain de l'amplificateur est ramené à $1$.

Concernant le cable reliant les points 1 du Jumper 1 (\textit{JP1}) et du Jumper 2 (\textit{JP2}), il s'agit en quelque sorte d'un cable de "`sécurité"' que l'on peut connecter afin que le signal ne passe pas par les filtres passe-haut et passe-bas. Cela pourrait nous être utile pour tester notre haut-parleur dans l'hypothèse ou un des filtres ne fonctionnerait pas.

Enfin, on peut aussi remarquer une boucle qui relie le point 1 du potentiomètre 3 (\textit{P3}) au reste du circuit. Cette boucle a simplement pour but d'éviter de laisser un cable "`dans le vide"', et donc d'éviter les signaux parasites (comme la tension dans les murs, etc).

\begin{figure}[h]
	\centering
	\includegraphics[scale=0.5]{img/bloc3.png}
	\caption{Schéma du bloc 3, chargé du réglage des aigus et des graves.}
	\label{bloc3}
\end{figure}

\subsection{Bloc 4 : l'amplificateur de puissance}

\begin{figure}[h]
	\centering
	\includegraphics[scale=0.6]{img/bloc4.png}
	\caption{Schéma du bloc 4, l'amplificateur de puissance.}
	\label{bloc4}
\end{figure}

Cette partie contient l'amplificateur de puissance (aussi appelé amplificateur audio). 
Cet amplificateur a pour but d'amplifier un signal électrique audio pour
permettre le fonctionnement d'un haut-parleur ou d'une enceinte acoustique.

Le nom amplificateur de puissance peut induire en erreur. En effet, comme toute amplificateur,
l'amplificateur de puissance agit sur la tension mais son impédance de sortie étant très faible,
il peut délivrer une grande puissance (ce qui lui vaut son nom).

Dans notre cas, l'amplificateur a un gain en tension de $50$.

Le premier condensateur (\textit{C1} sur la Figure \ref{bloc4}) sert à stabiliser la tension d'alimentation. Ce relativement "gros" condensateur
agit en quelque sorte comme un réservoir.

Le deuxième condensateur (\textit{COUT}) permet de ne laisser passer que les signaux alternatifs. Il bloque tous les courants continus parasites.

\end{document}