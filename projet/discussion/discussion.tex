\documentclass{article}

\usepackage[utf8]{inputenc}
\usepackage[T1]{fontenc}      
\usepackage[francais]{babel}
\usepackage{graphicx}
\usepackage{circuitikz}
\usepackage[squaren, Gray]{SIunits}
\usepackage{sistyle}
\usepackage[autolanguage]{numprint}
\usepackage{pgfplots}
\pgfplotsset{compat=1.9}
\usepackage{amsmath,amssymb,array}
\usepackage[top=2.5cm,bottom=2.5cm,right=2.5cm,left=2.5cm]{geometry}
\usepackage{url} 
\usepackage{tabularx}
\DeclareMathOperator{\dist}{d}
\newenvironment{abstract-fr}
{
	\begin{center}
		\textbf{Résumé} \\[0.5cm]
	\end{center}
}
{}

\newenvironment{abstract-en}
{
	\begin{center}
		\textbf{Summary} \\[0.5cm]
	\end{center}
}
{}
% New command pour la modélisation mécanique, tri à effectuer
\newcommand\fv[1]{{\bf #1}} % free vector
\newcommand\fvd[1]{\dot{\bf #1}} % free vector derivated
\newcommand\fvdd[1]{\ddot{\bf #1}} % free vector derivated
\newcommand\fvr[1]{\mathring{\bf #1}} % free vector relatively derivated
\newcommand\fvrr[1]{\overset{\circ\circ}{\bf #1}} % free vector relatively derivated
\newcommand\uv[1]{{\bf\hat{ #1}}} % unit vector
\newcommand\ui{{\bf\hat{I}}} % unit vector I
\newcommand\uj{{\bf\hat{J}}} % unit vector J
\newcommand\uk{{\bf\hat{K}}} % unit vector K
\newcommand\wrt[2]{\ensuremath{\tensor*[_{ #1}]{ #2}{}}} % With Respect To
\newcommand\wtr[3]{\ensuremath{\tensor*[_{ #1}]{ #2}{^{ #3}}}} % With Two Respect
\newcommand\omegaf{{\bm \omega}}
\newcommand\omegafr{\mathring{\bm \omega}}
\newcommand\omegafd{\dot{\bm \omega}}
\newcommand\omegaft{\tilde{\bm \omega}}
\newcommand\omegaftr{\mathring{\tilde{\bm \omega}}}
\newcommand\omegat{\tilde{\omega}}
\newcommand\omegatd{\tilde{\dot{\omega}}}
\newcommand\ine{{\bf I}}
\newcommand\st{{\bf L}}
\newcommand\pst{{\bf M}}
\newcommand\lm{{\bf N}}
\newcommand\am{{\bf H}}
\newcommand\amd{\dot{\am}}
\newcommand\fo{{\bf F}}
\newcommand\po{\mathcal{P}}
\newcommand\xg{\ensuremath{\fv{R}}}
\newcommand\xgd{\ensuremath{\fvd{R}}}
\newcommand\xgdd{\ensuremath{\fvdd{R}}}
\newcommand\dvec[1]{\dot{\vec{ #1}}}
\newcommand\ddvec[1]{\ddot{\vec{ #1}}}
\newcommand\qp{\dot{q}}
\newcommand\dqp{\Delta \dot{q}}
\usepackage{url} 
\usepackage{hyperref}
\hypersetup{
    colorlinks,
    citecolor=black,
    filecolor=black,
    linkcolor=black,
    urlcolor=black
}

\begin{document}

%Théorie et pratique
En théorie, un son devrait sortir du haut-parleur.  En pratique, il y a bien une fréquence
à la sortie de la plaquette mais le haut-parleur demeure aphone. Il y a donc manifestement 
un mauvais raccordement bobine fixe-bobine mobile.

%Mesure et modélisation
Les mesures réalisées sont moins précises que la simulation, mais nous obtenons tout de même 
des valeurs plus petite que celle attendues.  Par exemple, nous obtenons un champ magnétique
expérimental plus faible que calculé.  Cela est tout à fait normal vu que nous avons émis des
des hypothèses simplificatrices qui ne respectent pas tout à fait la réalité. Nous avions 
imposé la conservation du flux dans nos calculs, par exemple. Il est évident que des pertes de flux 
ont été subies, et tout le flux n'étais pas concentré dans l'entrefer, malgré que nous l'ayons minimisé.

%Système complet
Les points forts de notre système sont les suivants: la membrane, le caisson, et l'entrefer. La membrane a 
été réalisée avec du tissus et du papier, et ainsi éviter les pliages difficiles, mais tout de même garder 
une certaine rigidité.  Le caisson est également un fierté, étant donné qu'il a été pensé pour obtenir le
meilleur rapport qualité-prix. Enfin, nous avons pensé à réduire l'entrefer, pour maximiser le champ magnétique.
Le principal point faible de l'appareil est qu'il n'y a pas de son sortant du haut-parleur.

%Gestion du travail de groupe
Nous avons assez bien géré le groupe, l'ambiance dans le groupe était excellente.  Le travail était assez bien réparti
même si ce n'était pas possible que tout le monde ait la même charge de travail.  Grâce à des outils comme GitHub, Dropbox et 
\Latex, la communication pour la rédaction du rapport à été grandement facilitée.


% Just here to fix rapport_prejury.tex
\end{document}
