\documentclass{article}

\usepackage[utf8]{inputenc}
\usepackage[T1]{fontenc}      
\usepackage[francais]{babel}
\usepackage{graphicx}
\usepackage{circuitikz}
\usepackage[squaren, Gray]{SIunits}
\usepackage{sistyle}
\usepackage[autolanguage]{numprint}
\usepackage{pgfplots}
\pgfplotsset{compat=1.9}
\usepackage{amsmath,amssymb,array}
\usepackage[top=2.5cm,bottom=2.5cm,right=2.5cm,left=2.5cm]{geometry}
\usepackage{url} 
\usepackage{tabularx}
\DeclareMathOperator{\dist}{d}
\newenvironment{abstract-fr}
{
	\begin{center}
		\textbf{Résumé} \\[0.5cm]
	\end{center}
}
{}

\newenvironment{abstract-en}
{
	\begin{center}
		\textbf{Summary} \\[0.5cm]
	\end{center}
}
{}
% New command pour la modélisation mécanique, tri à effectuer
\newcommand\fv[1]{{\bf #1}} % free vector
\newcommand\fvd[1]{\dot{\bf #1}} % free vector derivated
\newcommand\fvdd[1]{\ddot{\bf #1}} % free vector derivated
\newcommand\fvr[1]{\mathring{\bf #1}} % free vector relatively derivated
\newcommand\fvrr[1]{\overset{\circ\circ}{\bf #1}} % free vector relatively derivated
\newcommand\uv[1]{{\bf\hat{ #1}}} % unit vector
\newcommand\ui{{\bf\hat{I}}} % unit vector I
\newcommand\uj{{\bf\hat{J}}} % unit vector J
\newcommand\uk{{\bf\hat{K}}} % unit vector K
\newcommand\wrt[2]{\ensuremath{\tensor*[_{ #1}]{ #2}{}}} % With Respect To
\newcommand\wtr[3]{\ensuremath{\tensor*[_{ #1}]{ #2}{^{ #3}}}} % With Two Respect
\newcommand\omegaf{{\bm \omega}}
\newcommand\omegafr{\mathring{\bm \omega}}
\newcommand\omegafd{\dot{\bm \omega}}
\newcommand\omegaft{\tilde{\bm \omega}}
\newcommand\omegaftr{\mathring{\tilde{\bm \omega}}}
\newcommand\omegat{\tilde{\omega}}
\newcommand\omegatd{\tilde{\dot{\omega}}}
\newcommand\ine{{\bf I}}
\newcommand\st{{\bf L}}
\newcommand\pst{{\bf M}}
\newcommand\lm{{\bf N}}
\newcommand\am{{\bf H}}
\newcommand\amd{\dot{\am}}
\newcommand\fo{{\bf F}}
\newcommand\po{\mathcal{P}}
\newcommand\xg{\ensuremath{\fv{R}}}
\newcommand\xgd{\ensuremath{\fvd{R}}}
\newcommand\xgdd{\ensuremath{\fvdd{R}}}
\newcommand\dvec[1]{\dot{\vec{ #1}}}
\newcommand\ddvec[1]{\ddot{\vec{ #1}}}
\newcommand\qp{\dot{q}}
\newcommand\dqp{\Delta \dot{q}}
\usepackage{url} 
\usepackage{hyperref}
\hypersetup{
    colorlinks,
    citecolor=black,
    filecolor=black,
    linkcolor=black,
    urlcolor=black
}

\begin{document}

\section{Méthode de recherche}

\subsection{La contre-réaction ou réaction négative}
Comme suggeré lors de la séance d'information sur la recherche bibliographique,
nous avons appliqué la méthode de l'entonnoir. Comme les boucles de contre-réaction 
sont directement liées aux amplificateurs, nous avons commencé nos recherches avec 
les termes plutôt généraux : \textit{amplificateurs} et \textit{amplifiers}. Nous 
nous avons ensuite associé à ces mots clés les termes plus précis : \textit{contre-réaction}
et \textit{negative feedback}.

Les différents ouvrages et documents que nous avons utilisés sont listés dans la bibliographie.

\subsection{La distorsion harmonique}

\paragraph{Choix du thème}
Le choix du thème n'a pas été chose aisée. Nous avons commencé par établir un brainstorming afin de réunir 
le plus d'idées possibles. Cependant, les thèmes proposés nous semblaient trop généraux que pour faire un vrai 
travail en profondeur tout en restant concis. Quelqu'un a finalement proposé la distorsion harmonique ; un 
terme visible sur les emballages de haut-parleurs. Nous avions également repéré ce terme dans la datasheet 
de l'amplificateur audio reçu pour le projet: une valeur de \numprint{0.2\%} était renseignée pour le THD 
(taux de distorsion harmonique). Curieux d'en apprendre plus sur ce terme presque méconnu, 
nous avons décidé de débuter notre travail de recherche là-dessus.

\paragraph{Recherche documentaire}
Etant donné que nous ne connaissions vraiment que très peu sur ce sujet et que nous 
devions le comprendre en profondeur, nous avons commencé par le terme général de "distorsion".
Une première recherche sur internet a permis de fixer les idées à propos de ce terme, et nous 
avons ensuite pu établir une liste de mot-clefs pour entamer réellement la recherche sur la 
distorsion harmonique. Nous avons appliqué la "technique de l'entonnoir", et nous avons finalement 
réuni assez d'informations que pour écrire ce rapport. Notons tout de même que c'est indiscutablement
en anglais que nous avons 
trouvé le plus d'informations. Nous avons gardé une trace de toutes les sources que nous avons 
consultées, et cela a rendu l'écriture de la bibliographie nettement plus facile.

% Just here to fix rapport_prejury.tex
\end{document}
