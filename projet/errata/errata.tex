% TODO
% Si vous remarquez des erreurs dans le rapport, signalez les ici.

1) 5.1, page 24
Dans la partie discussion, on dit qu'on a expérimentalement 8 cT, à la place des 7.54 cT calculés, et on explique 
cela par le fait qu'un a supposé tout le champ concentré dns l'enterfer... illogique du coup qu'on ait plus que ce qu'on a calculé.
Par contre je sais pas si on avait pris en compte le fait qu'on avait rapetissé l'entrefer dans nos calculs, donc ca vient peut
etre de là qu'on a un plus grand champ.

2) 2.3.1, page 12
- La perméabilité relative de l'air n'a pas d'unité, or dans le rapport il est mis H/m.

3) 2.2.1, page 7
Dans la "Recherche de la solution homogène", il est indiqué "A est une constante appartenant
à l'ensemble des réels", je pense que A appartient en fait à l'ensemble des complexes.

4) 2.4.2, page 15
Dans la résolution de l'équation différentielle du mouvement, il est noté après le calcul
des racines du polynôme caractéristiques : "Nous avons donc, en ne gardant que la partie réelle".
Il faut supprimer la partie "en ne gardant que la partie réelle".

5) 6.1.1, page 29
Il est écrit "Pour trouver la fréquence d'intersection entre les deux droites, nous résolvons le système".
On recherche en fait la fréquence de coupure, qui correspond à l'abscisse à laquelle s'intersecte les
deux droites, c'est plus précis. De plus, le système a déjà été résolu. Ici on recherche simplement la
solution de l'équation -1.96 log(x) + 9.84 = 2.5.

Orthographe
-----------
Sinon, il restait qqes fautes de frappe/orthographe (shame on me) Mais j'imagine qu'on s'en fout
- 2.3.1 : auX courantS de Foucault (il y en a plusieurs non?)
- 2.3.3 : comme cette bobine mobile est traverséE par... (rajouter le E)
- 2.4.2 : juuste avant qu'elle ne commenceR à retourner... (backer le r)
- 4.1 : résistanCe (rajouter le C)
- 5.1 : nous sommes assez fierS (rajouter le S)
- 6.1.1 : le \vec n'est pas bien passé, il est écrit 'ecf2' (vers les 3/4 de la page)
