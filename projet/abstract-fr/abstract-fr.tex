\documentclass{article}

\usepackage[utf8]{inputenc}
\usepackage[T1]{fontenc}      
\usepackage[francais]{babel}
\usepackage{graphicx}
\usepackage{circuitikz}
\usepackage[squaren, Gray]{SIunits}
\usepackage{sistyle}
\usepackage[autolanguage]{numprint}
\usepackage{pgfplots}
\pgfplotsset{compat=1.9}
\usepackage{amsmath,amssymb,array}
\usepackage[top=2.5cm,bottom=2.5cm,right=2.5cm,left=2.5cm]{geometry}
\usepackage{url} 
\usepackage{tabularx}
\DeclareMathOperator{\dist}{d}
\newenvironment{abstract-fr}
{
	\begin{center}
		\textbf{Résumé} \\[0.5cm]
	\end{center}
}
{}

\newenvironment{abstract-en}
{
	\begin{center}
		\textbf{Summary} \\[0.5cm]
	\end{center}
}
{}
% New command pour la modélisation mécanique, tri à effectuer
\newcommand\fv[1]{{\bf #1}} % free vector
\newcommand\fvd[1]{\dot{\bf #1}} % free vector derivated
\newcommand\fvdd[1]{\ddot{\bf #1}} % free vector derivated
\newcommand\fvr[1]{\mathring{\bf #1}} % free vector relatively derivated
\newcommand\fvrr[1]{\overset{\circ\circ}{\bf #1}} % free vector relatively derivated
\newcommand\uv[1]{{\bf\hat{ #1}}} % unit vector
\newcommand\ui{{\bf\hat{I}}} % unit vector I
\newcommand\uj{{\bf\hat{J}}} % unit vector J
\newcommand\uk{{\bf\hat{K}}} % unit vector K
\newcommand\wrt[2]{\ensuremath{\tensor*[_{ #1}]{ #2}{}}} % With Respect To
\newcommand\wtr[3]{\ensuremath{\tensor*[_{ #1}]{ #2}{^{ #3}}}} % With Two Respect
\newcommand\omegaf{{\bm \omega}}
\newcommand\omegafr{\mathring{\bm \omega}}
\newcommand\omegafd{\dot{\bm \omega}}
\newcommand\omegaft{\tilde{\bm \omega}}
\newcommand\omegaftr{\mathring{\tilde{\bm \omega}}}
\newcommand\omegat{\tilde{\omega}}
\newcommand\omegatd{\tilde{\dot{\omega}}}
\newcommand\ine{{\bf I}}
\newcommand\st{{\bf L}}
\newcommand\pst{{\bf M}}
\newcommand\lm{{\bf N}}
\newcommand\am{{\bf H}}
\newcommand\amd{\dot{\am}}
\newcommand\fo{{\bf F}}
\newcommand\po{\mathcal{P}}
\newcommand\xg{\ensuremath{\fv{R}}}
\newcommand\xgd{\ensuremath{\fvd{R}}}
\newcommand\xgdd{\ensuremath{\fvdd{R}}}
\newcommand\dvec[1]{\dot{\vec{ #1}}}
\newcommand\ddvec[1]{\ddot{\vec{ #1}}}
\newcommand\qp{\dot{q}}
\newcommand\dqp{\Delta \dot{q}}
\usepackage{url} 
\usepackage{hyperref}
\hypersetup{
    colorlinks,
    citecolor=black,
    filecolor=black,
    linkcolor=black,
    urlcolor=black
}

\begin{document}

\begin{abstract-fr}
Le haut-parleur est un outil permettant de transformer un signal électrique en un signal sonore.  Grâce à une bobine fixe, la bobine mobile peut osciller et faire vibrer la membrane à laquelle elle est attachée, ceci crée une onde sonore.
De nombreuses phases de tests ont été effectuées pour obtenir un haut-parleur fonctionnel, malgré les calculs mathématiques nous avons dû faire des hypothèses simplificatrices et donc seuls les vrais tests reflètent la réalité.
Dans ce document, les calculs et les idées nécessaires à a création d’un tel objet sont présentés.  
Notre haut-parleur est capable de produire un son qui a une fréquence entre \SI{500}{\hertz} et \SI{5000}{\hertz}, a une puissance de \SI{2.5}{\watt}, avec une membrane de \SI{0.17}{\meter} de diamètre, une bobine fixe de \numprint{450} spires et une mobile de \numprint{100}. Avec la prise jack qui permet de relier le haut-parleur à un Gsm ou un mp3, nous pouvons obtenir un son réglable en intensité.
\end{abstract-fr}

% Just here to fix rapport_prejury.tex
\end{document}
