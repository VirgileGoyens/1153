\documentclass{article}

\usepackage[utf8]{inputenc}
\usepackage[T1]{fontenc}      
\usepackage[francais]{babel}
\usepackage{graphicx}
\usepackage{circuitikz}
\usepackage[squaren, Gray]{SIunits}
\usepackage{sistyle}
\usepackage[autolanguage]{numprint}
\usepackage{pgfplots}
\pgfplotsset{compat=1.9}
\usepackage{amsmath,amssymb,array}
\usepackage[top=2.5cm,bottom=2.5cm,right=2.5cm,left=2.5cm]{geometry}
\usepackage{url} 
\usepackage{tabularx}
\DeclareMathOperator{\dist}{d}
\newenvironment{abstract-fr}
{
	\begin{center}
		\textbf{Résumé} \\[0.5cm]
	\end{center}
}
{}

\newenvironment{abstract-en}
{
	\begin{center}
		\textbf{Summary} \\[0.5cm]
	\end{center}
}
{}
% New command pour la modélisation mécanique, tri à effectuer
\newcommand\fv[1]{{\bf #1}} % free vector
\newcommand\fvd[1]{\dot{\bf #1}} % free vector derivated
\newcommand\fvdd[1]{\ddot{\bf #1}} % free vector derivated
\newcommand\fvr[1]{\mathring{\bf #1}} % free vector relatively derivated
\newcommand\fvrr[1]{\overset{\circ\circ}{\bf #1}} % free vector relatively derivated
\newcommand\uv[1]{{\bf\hat{ #1}}} % unit vector
\newcommand\ui{{\bf\hat{I}}} % unit vector I
\newcommand\uj{{\bf\hat{J}}} % unit vector J
\newcommand\uk{{\bf\hat{K}}} % unit vector K
\newcommand\wrt[2]{\ensuremath{\tensor*[_{ #1}]{ #2}{}}} % With Respect To
\newcommand\wtr[3]{\ensuremath{\tensor*[_{ #1}]{ #2}{^{ #3}}}} % With Two Respect
\newcommand\omegaf{{\bm \omega}}
\newcommand\omegafr{\mathring{\bm \omega}}
\newcommand\omegafd{\dot{\bm \omega}}
\newcommand\omegaft{\tilde{\bm \omega}}
\newcommand\omegaftr{\mathring{\tilde{\bm \omega}}}
\newcommand\omegat{\tilde{\omega}}
\newcommand\omegatd{\tilde{\dot{\omega}}}
\newcommand\ine{{\bf I}}
\newcommand\st{{\bf L}}
\newcommand\pst{{\bf M}}
\newcommand\lm{{\bf N}}
\newcommand\am{{\bf H}}
\newcommand\amd{\dot{\am}}
\newcommand\fo{{\bf F}}
\newcommand\po{\mathcal{P}}
\newcommand\xg{\ensuremath{\fv{R}}}
\newcommand\xgd{\ensuremath{\fvd{R}}}
\newcommand\xgdd{\ensuremath{\fvdd{R}}}
\newcommand\dvec[1]{\dot{\vec{ #1}}}
\newcommand\ddvec[1]{\ddot{\vec{ #1}}}
\newcommand\qp{\dot{q}}
\newcommand\dqp{\Delta \dot{q}}
\usepackage{url} 
\usepackage{hyperref}
\hypersetup{
    colorlinks,
    citecolor=black,
    filecolor=black,
    linkcolor=black,
    urlcolor=black
}

\begin{document}

\begin{abstract-fr}
% Contexe et tâche
Dans le cadre du cours \textit{Projet 2}, il nous a été demandé
de concevoir un haut-parleur que l'on puisse connecter à un smartphone
par le biais d'une prise jack.
% Besoin
Un haut-parleur est un outil permettant de transformer un signal 
électrique en un son. Grâce à un électro-aimant constituée d'une 
bobine fixe dans laquelle passe du courant, une bobine mobile 
oscille et fait vibrer la membrane à laquelle elle est attachée. 
Cette oscillation, qui dépend de la fréquence du signal électrique, 
crée une onde sonore.
% Objet
Lors des différentes étapes de modélisations mathématiques et physiques
de ce projet, nous avons utiliser de nombreuses hypothèses simplificatrices. 
Ces hypothèses simplificatrices aboutissent à des modèles simplifiés de la
réalité. Cependant, les approximations effectués sont en général assez proche
des mesures faites en laboratoires comme le montre la Figure \ref{comparaison}.
Ce document reprend en détails la modélisation de plusieurs parties d'un haut-parleur.
% Conclusion et perspective
Tout au long de ce projet, nous avons appris à modéliser des situations réelles
complexes de manière simplifiée en utilisant des hypothèses simplificatrices

\end{abstract-fr}

\begin{figure}
	\centering
	
	\caption{}
\end{figure}

% Bar charts de l'erreur exprimée en pourcentage
%\begin{figure}[ht]
	%\centering
    %\begin{tikzpicture}
        %\begin{axis}[
            %symbolic x coords={1, 2, 3, 4, 5},
            %xtick = data
          %]
            %\addplot[ybar,fill=bblue] coordinates {
                %(1, 12)
                %(2, 6)
                %(3, 9)
								%(4, 5)
								%(5, 3)
            %};
        %\end{axis}
    %\end{tikzpicture}
		%\caption{Comparaison entre le modèle et la réalité (erreure exprimée en
		%pourcentage). Comparison between model and reality (error in percentage).}
		%\label{comparaison}
%\end{figure}

% Just here to fix rapport_prejury.tex
\end{document}
