\documentclass{article}

\usepackage[utf8]{inputenc} 
\usepackage[T1]{fontenc}      
\usepackage[francais]{babel} 
\usepackage{graphicx}


\begin{document}

\section{Recherche documentaire: la distorsion harmonique}

La distortion est un critère de qualité en ce qui concerne les haut-parleurs.
Dans le soucis de construire un dispositif de qualité, nous avons décidé de 
nous informer sur la distorsion harmonique, un concept que nous ne connaissions
que de nom.
Ce document est structuré comme suit: Nous parlerons tout d'abord de la méthode 
de recherche que nous avons adoptée, pour ensuite aborder la notion  de distorsion 
en général, et finalement décrire la distorsion harmonique, ses causes, ses effets,
et les moyens de diminution.

\paragraph{Méthode de recherche}
Etant donné que nous ne connaissions vraiment que très peu sur ce sujet et que nous 
le devions comprendre en profondeur, nous avons commencé par le terme général de "distorsion". 
Une première recherche sur internet a permis de fixer les idées à propos de ce thème, et nous 
avons ensuite pu établir une liste de mot-clefs pour entamer réellement la recherche sur la 
distorsion harmonique. Nous avons appliqué la "technique de l'entonnoir", et nous avons finalement 
réuni assez d'informations que pour écrire ce rapport. L'encyclopédie \textit{Universalis} nous a 
été d'une grande aide. Notons tout de même que c'est indiscutablement en anglais que nous avons 
trouvé le plus d'informations. Nous avons gardé une trace de toutes les sources que nous avons 
consultées, et cela a rendu l'écriture de la bibliographie nettement plus facile.

\subsection{Définition}
Commençons tout d'abord par comprendre la notion de distorsion du son: par définition, c'est
une transformation du signal audio par rapport à celui de sortie. Ces distorsions ne sont pas vraiment souhaitées, étant donné
que le signal en est déformé. Cependant, certains audiophiles en tirent avantage, vu que que quelques
transformations peuvent mener à un son plus chaud et agréable.

\paragraph{La distorsion harnomique}
La distorsion harmonique joue sur la superposition de différentes fréquences:
la fréquence fondamentale et ses harmoniques. Un haut-parleur parfait émettrait seulement la fréquence fondamentale, sans les harmoniques, qui sont donc des "parasites".
On parle d'harmonique pour désigner les multiples entiers de la fréquence fondamentale.
Par exemple, la seconde harmonique d'une fréquence de 50 Hz vaut 100Hz, la troisième 150Hz, etc.
Elles s'organisent en deux familles : 
Les harmoniques paires sont les moins incommodantes, étant donné qu'elles représentent la même note, mais à quelques octaves de différence.
Les harmoniques impaires, elles, sont plus gênantes étant donné que la note est différente.



\begin{figure}[h]
\centering
\includegraphics[scale=0.6]{image2.png}
\caption{Superposition d'une fréquence fondamentale et de ses premières hamoniques.}
\label{Superposition d'une fréquence fondamentale et de ses premières hamoniques.}
\end{figure}



\subsection{Causes}
Une des causes principales de la distorsion harmonique est le haut parleur, qui ajoute des distorsions au signal.
Les vibrations dans l'enceinte sont également uen source de distorsion du signal.


\subsection{Conséquences}
Lorsque les fréquences sont correctement ajustées, l'ajout d'harmoniques peut rendre le son plus chaud et 
agréable à écouter. C'est pour cela que certains la recherchent.
TIMBRE
Ces distorsions sont parfois peu audibles, et un certain pourcentage est tout à fait acceptable puisuqe notre
oreille y est tolérante.
TAUX DE DISTORTION


\subsection{Solutions}
Pour éviter les mauvaises distorsion, ou tout simplement pour émettre un son pur et exact, il existe différentes
solutions.
Certains auiophiles cherchent aussi à accentuer ces distortions. Les différents  moyens d'y parvenir sont...


\bibliographystyle{plain}
\bibliography{source}

\end{document}
