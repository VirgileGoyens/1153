\documentclass{article}

\usepackage[utf8]{inputenc}
\usepackage[T1]{fontenc}      
\usepackage[francais]{babel}
\usepackage{graphicx}
\usepackage{circuitikz}
\usepackage[squaren, Gray]{SIunits}
\usepackage{sistyle}
\usepackage[autolanguage]{numprint}
\usepackage{pgfplots}
\pgfplotsset{compat=1.9}
\usepackage{amsmath,amssymb,array}
\usepackage[top=2.5cm,bottom=2.5cm,right=2.5cm,left=2.5cm]{geometry}
\usepackage{url} 
\usepackage{tabularx}
\DeclareMathOperator{\dist}{d}
\newenvironment{abstract-fr}
{
	\begin{center}
		\textbf{Résumé} \\[0.5cm]
	\end{center}
}
{}

\newenvironment{abstract-en}
{
	\begin{center}
		\textbf{Summary} \\[0.5cm]
	\end{center}
}
{}
% New command pour la modélisation mécanique, tri à effectuer
\newcommand\fv[1]{{\bf #1}} % free vector
\newcommand\fvd[1]{\dot{\bf #1}} % free vector derivated
\newcommand\fvdd[1]{\ddot{\bf #1}} % free vector derivated
\newcommand\fvr[1]{\mathring{\bf #1}} % free vector relatively derivated
\newcommand\fvrr[1]{\overset{\circ\circ}{\bf #1}} % free vector relatively derivated
\newcommand\uv[1]{{\bf\hat{ #1}}} % unit vector
\newcommand\ui{{\bf\hat{I}}} % unit vector I
\newcommand\uj{{\bf\hat{J}}} % unit vector J
\newcommand\uk{{\bf\hat{K}}} % unit vector K
\newcommand\wrt[2]{\ensuremath{\tensor*[_{ #1}]{ #2}{}}} % With Respect To
\newcommand\wtr[3]{\ensuremath{\tensor*[_{ #1}]{ #2}{^{ #3}}}} % With Two Respect
\newcommand\omegaf{{\bm \omega}}
\newcommand\omegafr{\mathring{\bm \omega}}
\newcommand\omegafd{\dot{\bm \omega}}
\newcommand\omegaft{\tilde{\bm \omega}}
\newcommand\omegaftr{\mathring{\tilde{\bm \omega}}}
\newcommand\omegat{\tilde{\omega}}
\newcommand\omegatd{\tilde{\dot{\omega}}}
\newcommand\ine{{\bf I}}
\newcommand\st{{\bf L}}
\newcommand\pst{{\bf M}}
\newcommand\lm{{\bf N}}
\newcommand\am{{\bf H}}
\newcommand\amd{\dot{\am}}
\newcommand\fo{{\bf F}}
\newcommand\po{\mathcal{P}}
\newcommand\xg{\ensuremath{\fv{R}}}
\newcommand\xgd{\ensuremath{\fvd{R}}}
\newcommand\xgdd{\ensuremath{\fvdd{R}}}
\newcommand\dvec[1]{\dot{\vec{ #1}}}
\newcommand\ddvec[1]{\ddot{\vec{ #1}}}
\newcommand\qp{\dot{q}}
\newcommand\dqp{\Delta \dot{q}}
\usepackage{url} 
\usepackage{hyperref}
\hypersetup{
    colorlinks,
    citecolor=black,
    filecolor=black,
    linkcolor=black,
    urlcolor=black
}

\begin{document}

\section{Fonctionnement général}

Dans cette première section, nous allons survoler le fonctionnement général de notre haut-parleur, ainsi 
qu'introduire certains concepts physiques clefs. Les sections suivantes détailleront tout cela plus en 
profondeur.

\begin{figure}[ht!] 
\centering 
\includegraphics[scale=0.25]{vue_d_ensemble.png} 
\caption{Vue éclatée du haut-parleur} 
\label{hp-scheme} 
\end{figure}

\paragraph{}

\paragraph{Support en "E" et bobine fixe}
Un haut-parleur est en général constitué d'un aimant permanent et d'une bobine mobile faisant vibrer une 
membrane. Malheureusement, nous ne disposions pas d'un tel aimant ; nous avons donc été contraints de créer
notre propre électroaimant. Ce dernier est constitué d'un support en forme de "E", un matériau ferromagnétique, et 
d'une bobine fixe parcourue par un courant continu. La caractéristique principale d'un tel matériau est sa 
faculté à amplifier un champ magnétique, aussi appelée perméabilité magnétique \cite{wiki-perm-magn}
Lorsque le courant traverse la bobine de cuivre, un champ magnétique est formé. Nous supposerons par la 
suite que tout le flux passe dans les branches du "E", sans perte. Ce dispositif va se comporter comme un
aimant.
% loi d'Ampère?

\paragraph{Bobine mobile et membrane}
Le champ crée par l'électroaimant est intercepté par la bobine dite \textit{mobile}. Lorsqu'une telle bobine
de longueur $L$, traversée par un courant $I$, est plongée dans un champ magnétique, une force $F$ est exercée
sur celle-ci. Cette dernière est appelée \texit{force de Laplace}. Elle s'exprime comme suit:

$$\vec{F} = I\vec{l}\times{\vec{B}}$$ 

Cette force va faire bouger la bobine mobile, et conséquemment faire vibrer la membrane. Un son est alors 
produit. Les réglages du volume, des aigus et des graves se font sur le circuit imprimé, qui est lui-même relié à la
bobine mobile. L'annexe "Analyse séquentielle du circuit" explicite tout cela en détail.

% Just here to fix rapport_prejury.tex
\end{document}