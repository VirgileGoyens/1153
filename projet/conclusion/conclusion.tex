\documentclass{article}

\usepackage[utf8]{inputenc}
\usepackage[T1]{fontenc}      
\usepackage[francais]{babel}
\usepackage{graphicx}
\usepackage{circuitikz}
\usepackage[squaren, Gray]{SIunits}
\usepackage{sistyle}
\usepackage[autolanguage]{numprint}
\usepackage{pgfplots}
\pgfplotsset{compat=1.9}
\usepackage{amsmath,amssymb,array}
\usepackage[top=2.5cm,bottom=2.5cm,right=2.5cm,left=2.5cm]{geometry}
\usepackage{url} 
\usepackage{tabularx}
\DeclareMathOperator{\dist}{d}
\newenvironment{abstract-fr}
{
	\begin{center}
		\textbf{Résumé} \\[0.5cm]
	\end{center}
}
{}

\newenvironment{abstract-en}
{
	\begin{center}
		\textbf{Summary} \\[0.5cm]
	\end{center}
}
{}
% New command pour la modélisation mécanique, tri à effectuer
\newcommand\fv[1]{{\bf #1}} % free vector
\newcommand\fvd[1]{\dot{\bf #1}} % free vector derivated
\newcommand\fvdd[1]{\ddot{\bf #1}} % free vector derivated
\newcommand\fvr[1]{\mathring{\bf #1}} % free vector relatively derivated
\newcommand\fvrr[1]{\overset{\circ\circ}{\bf #1}} % free vector relatively derivated
\newcommand\uv[1]{{\bf\hat{ #1}}} % unit vector
\newcommand\ui{{\bf\hat{I}}} % unit vector I
\newcommand\uj{{\bf\hat{J}}} % unit vector J
\newcommand\uk{{\bf\hat{K}}} % unit vector K
\newcommand\wrt[2]{\ensuremath{\tensor*[_{ #1}]{ #2}{}}} % With Respect To
\newcommand\wtr[3]{\ensuremath{\tensor*[_{ #1}]{ #2}{^{ #3}}}} % With Two Respect
\newcommand\omegaf{{\bm \omega}}
\newcommand\omegafr{\mathring{\bm \omega}}
\newcommand\omegafd{\dot{\bm \omega}}
\newcommand\omegaft{\tilde{\bm \omega}}
\newcommand\omegaftr{\mathring{\tilde{\bm \omega}}}
\newcommand\omegat{\tilde{\omega}}
\newcommand\omegatd{\tilde{\dot{\omega}}}
\newcommand\ine{{\bf I}}
\newcommand\st{{\bf L}}
\newcommand\pst{{\bf M}}
\newcommand\lm{{\bf N}}
\newcommand\am{{\bf H}}
\newcommand\amd{\dot{\am}}
\newcommand\fo{{\bf F}}
\newcommand\po{\mathcal{P}}
\newcommand\xg{\ensuremath{\fv{R}}}
\newcommand\xgd{\ensuremath{\fvd{R}}}
\newcommand\xgdd{\ensuremath{\fvdd{R}}}
\newcommand\dvec[1]{\dot{\vec{ #1}}}
\newcommand\ddvec[1]{\ddot{\vec{ #1}}}
\newcommand\qp{\dot{q}}
\newcommand\dqp{\Delta \dot{q}}
\usepackage{url} 
\usepackage{hyperref}
\hypersetup{
    colorlinks,
    citecolor=black,
    filecolor=black,
    linkcolor=black,
    urlcolor=black
}

\begin{document}

En conclusion, nous avons compris le fonctionnement d’un haut-parleur,  nous avons appris à travailler en groupe, de façon organisée et responsable et nous avons essayé de fabriquer notre propre haut-parleur.  En toute objectivité, aucune musique ne sortait de notre haut-parleur.  Cependant nous voyions clairement grâce à l’oscilloscope la fréquence de la musique à la sortie du circuit imprimé.
Si l’on peut dire que la fabrication du haut-parleur est un échec, on peut néanmoins voir de nombreux points positifs concernant notre projet.  Nous avons compris et assimilé le fonctionnement théorique d’un haut-parleur, nous avons modélisé un haut-parleur avec des règles physiques et mathématiques, nous avons également su caractérisé la puissance, la taille, toutes les composantes interne d’un haut-parleur. 
Pour terminer, les concepts mathématiques et physiques ont bien été assimilés mais en pratique, le haut-parleur ne fonctionnait pas comme nous le souhaitions.  Il y a eu des problèmes entre la théorie et la pratique malgré les tests que nous avons fait pour que le haut-parleur fonctionne aussi bien qu’il devrait théoriquement le faire. 

% Just here to fix rapport_prejury.tex
\end{document}
