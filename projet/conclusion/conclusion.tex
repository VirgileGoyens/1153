\documentclass{article}

\usepackage[utf8]{inputenc}
\usepackage[T1]{fontenc}      
\usepackage[francais]{babel}
\usepackage{graphicx}
\usepackage{circuitikz}
\usepackage[squaren, Gray]{SIunits}
\usepackage{sistyle}
\usepackage[autolanguage]{numprint}
\usepackage{pgfplots}
\pgfplotsset{compat=1.9}
\usepackage{amsmath,amssymb,array}
\usepackage[top=2.5cm,bottom=2.5cm,right=2.5cm,left=2.5cm]{geometry}
\usepackage{url} 
\usepackage{tabularx}
\DeclareMathOperator{\dist}{d}
\newenvironment{abstract-fr}
{
	\begin{center}
		\textbf{Résumé} \\[0.5cm]
	\end{center}
}
{}

\newenvironment{abstract-en}
{
	\begin{center}
		\textbf{Summary} \\[0.5cm]
	\end{center}
}
{}
% New command pour la modélisation mécanique, tri à effectuer
\newcommand\fv[1]{{\bf #1}} % free vector
\newcommand\fvd[1]{\dot{\bf #1}} % free vector derivated
\newcommand\fvdd[1]{\ddot{\bf #1}} % free vector derivated
\newcommand\fvr[1]{\mathring{\bf #1}} % free vector relatively derivated
\newcommand\fvrr[1]{\overset{\circ\circ}{\bf #1}} % free vector relatively derivated
\newcommand\uv[1]{{\bf\hat{ #1}}} % unit vector
\newcommand\ui{{\bf\hat{I}}} % unit vector I
\newcommand\uj{{\bf\hat{J}}} % unit vector J
\newcommand\uk{{\bf\hat{K}}} % unit vector K
\newcommand\wrt[2]{\ensuremath{\tensor*[_{ #1}]{ #2}{}}} % With Respect To
\newcommand\wtr[3]{\ensuremath{\tensor*[_{ #1}]{ #2}{^{ #3}}}} % With Two Respect
\newcommand\omegaf{{\bm \omega}}
\newcommand\omegafr{\mathring{\bm \omega}}
\newcommand\omegafd{\dot{\bm \omega}}
\newcommand\omegaft{\tilde{\bm \omega}}
\newcommand\omegaftr{\mathring{\tilde{\bm \omega}}}
\newcommand\omegat{\tilde{\omega}}
\newcommand\omegatd{\tilde{\dot{\omega}}}
\newcommand\ine{{\bf I}}
\newcommand\st{{\bf L}}
\newcommand\pst{{\bf M}}
\newcommand\lm{{\bf N}}
\newcommand\am{{\bf H}}
\newcommand\amd{\dot{\am}}
\newcommand\fo{{\bf F}}
\newcommand\po{\mathcal{P}}
\newcommand\xg{\ensuremath{\fv{R}}}
\newcommand\xgd{\ensuremath{\fvd{R}}}
\newcommand\xgdd{\ensuremath{\fvdd{R}}}
\newcommand\dvec[1]{\dot{\vec{ #1}}}
\newcommand\ddvec[1]{\ddot{\vec{ #1}}}
\newcommand\qp{\dot{q}}
\newcommand\dqp{\Delta \dot{q}}
\usepackage{url} 
\usepackage{hyperref}
\hypersetup{
    colorlinks,
    citecolor=black,
    filecolor=black,
    linkcolor=black,
    urlcolor=black
}

\begin{document}

\section{Conclusion}

Nous voici finalement arrivés au terme de notre projet. Il y a douze semaines de cela, aucun 
d'entre nous ne connaissait le fonctionnement d'un haut-parleur. Aucun d'entre nous ne savait vraiment utiliser 
le matériel d'un laboratoire. Aucun de nous ne maîtrisait entièrement ne fût-ce qu'une parcelle de ce que nous 
avons appris. Aujourd'hui, nous pouvons nous targuer d'avoir énormément progressé; que ce soit d'un point de
vue scientifique, mathématique, ou organisationnel. Il est temps maintenant de prendre du recul, et jeter un 
regard critique sur ce que nous avons accompli.

% bilan aux niveaux scientifiques et techniques
\paragraph{Formation scientifique et technique}
Ce projet nous a permis de mettre en pratique de nombreuses notions abordées aux cours de mathématiques et
de physique, que ce soit au premier ou au second quadrimestre. 
Au niveau scientifique, nous avons abordé différents concepts physiques clefs comme la loi d'\textsc{Ampère},
la force de \textsc{Laplace}, la loi de \textsc{Hoocke}, mais également tout ce qui concerne la 
magnétostatique dans le vide et la matière, les matériaux magnétiques, les filtres passe-haut et passe-bas,...
De plus, nous avons été amenés à nous renseigner sur la distorsion harmonique et la contre-réaction, pour 
finalement rédiger un résumé de ce que nous avions appris.
Au niveau technique, nous avons compris et assimilé le fonctionnement théorique d’un haut-parleur, nous avons
étudié le circuit imprimé et ses composants, nous avons utilisé les fiches de spécifications de certains 
éléments et matériaux pour en tirer ce qui nous intéressait, nous avons étudié la mécanique de la membrane
ainsi que bien d'autres aspects encore.
À chaque question que nous nous sommes posée, nous nous sommes efforcés d'apporter des réponses techniques 
de qualité.

% bilan au niveau de la gestion: forces et faiblesses  + décisions importantes
\paragraph{Organisation et travail de groupe}
Étant donné que nous avions déjà participé à un projet au premier quadrimestre, nous avons pu en exploiter 
notre expérience.  Pour la plupart des membres du groupe, le projet était plus structuré dans notre groupe
actuel que dans les anciens. Nous avons effectivement essayé de faire un juste partage des tâches, et chaque
membre a pu apporter sa contribution. Un autre point à relever est le fait que le groupe est resté soudé 
pendant toute la durée du projet. Tout le monde était présent aux séances sauf en cas de force majeure. 
Cependant, nous n'avons pas assez privilégié les réunions réelles, étant donné que nous travaillions de notre
côté pour seulement mettre en commun par la suite. Par conséquent, le même travail était parfois réalisé
plusieurs fois. Un meilleur rendement aurait fait avancer le projet plus rapidement et plus intelligemment.
Cependant, toutes les grandes décisions telles que le fait de se focaliser sur un seul haut-parleur lorsque 
nous avons commencé à manquer de temps, ont été prises en groupe.

% outils pour le travail
Un dernier point important concernant le travail de groupe est l'utilisation d'outils pour la mise en commun.
Le fait que nous utilisions les mêmes outils a facilité l’échange de documents et d’informations. 
Par exemple, nous avons fait l’effort d’apprendre \LaTeX pour écrire notre rapport; nous avons également créé
une Dropbox ainsi qu’un compte GitHub où tous les documents étaient modifiables à tout moment du jour et de 
la nuit. Lorsqu'un changement était effectué, tous les membres du groupe en étaient avertis. 
Notons tout de même que nous n'avons pas négligé la réunion physique puisqu'elle reste 
le meilleur moyen de communiquer. 
% planning utilisé pr les échéances + commentaire
Grâce au planning réalisé lors du pré-jury, nous avons su avancer dans le projet de manière organisée et 
claire. Il nous a bien servi pour acquérir une vision structurée du projet, des échéances
et des délivrables. Malgré quelques écarts, nous nous sommes assez bien tenus au plan. Des problèmes répétitifs 
au niveau du circuit imprimé ont cependant retardé notre réalisation, et c'est ainsi que notre haut-parleur 
n'est pas tout à fait fonctionnel au final. À part cela, le cahier des charges a été respecté dans son ensemble.

% cohérence entre cdc et ccls
En conclusion, même si le haut-parleur ne fonctionnait pas comme nous le souhaitions, les concepts 
mathématiques et physiques ont été tout à fait assimilés. Notre groupe est resté solidaire durant tout 
le quadrimestre; prenant le temps de s'assurer de la compréhension de chacun. Ce projet nous aura donc été
grandement profitable, et c'est avec une grande fierté que nous y apportons le point final.

% Just here to fix rapport_prejury.tex
\end{document}
