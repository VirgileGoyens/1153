\documentclass{article}

\usepackage[utf8]{inputenc}
\usepackage[T1]{fontenc}
\usepackage[francais]{babel}
\usepackage{amsmath,amssymb,array}
 
\begin{document}
 
\section{Approximation de la fréquence de coupure}

\subsection{Pour le filtre passe-bas}

\subsubsection{Equation de la droite horizontale} % A refaire avec la méthode d'approximation
Nous savons que la droite horizontale a une valeur initiale de 2.5 V et donc \[y=2.5\]

\subsubsection{Equation de la droite diagonale}

Nous savons que l'équation de la droite est de type $y=a*x+b$
\\
Mais pour cette situation-ci, nous allons utiliser une base logarithmique pour la pente.  En effet, les différentes fréquences utilisées sont tellement éloignées les unes des autres que le graphique serait gigantesque et la pente diagonale serait en fait une courbe.  Ce qui donne: $y=a*\log{x}+b$

Voici 3 résultats mesurés en laboratoire.

\bigbreak
\\
\begin{tabular}{|c|c|c|}
\hline
V_c & f & log{ f} \\
\hline
1.7 & 16000 & 4.204\\
\hline
1.55 & 18000 & 4.255\\
\hline
1.45 & 20000 & 4.301 \\
\hline
\end{tabular}

\bigbreak
Des maintenant les fréquences sont exprimées en base logarithmique et nous obtenons la matrice suivante :
\bigbreak

$$
\begin{pmatrix}  
 4.204 & 1\\
 4.255 & 1 \\
 4.301 & 1 
\end{pmatrix}
\begin{pmatrix}  
a\\
b
\end{pmatrix}
=
\begin{pmatrix}  
1.7\\
1.55\\
1.45
\end{pmatrix}
$$

\bigbreak

Ce qui nous donne les vecteurs suivants:

\[e_1=( \frac{1}{\sqrt[]{3}} \frac{1}{\sqrt[]{3}} \frac{1}{\sqrt[]{3}})\]

\\ et

\\
\[e_2=( -0.68, 0.03, 0.73)\]

\bigbreak
Ce qui nous donne une projection de 
$$
\begin{pmatrix}  
1.6\\
1.5\\
1.4
\end{pmatrix}$$
$$

\bigbreak
Nous en déduisons la valeur des coefficients a et b:  
\[ a =-1.96 \]
\[ b= 9.84 \]

$$\fbox{y= -1.96 \timeslog{x} +9.84}$$

\bigbreak
Pour trouver la fréquence d'intersection entre les deux droites $$y=2.5$$ et $$y= -1.96 \times log{x} +9.84$$ nous égalisons les y et nous trouvons $$\fbox{x=5557.7 Hz$$} 

\\
Cela nous semble correct car en théorie nous devons arriver à une valeur f tel que $$f=\frac{1}{2\times \pi\times R\times C}$$
avec $R=7.5+50=57.5 ohms$ et $C=470\times 10^{-9} F$  Notre valeur théorique de la fréquence est donc $$f=5889.2 Hz$$


\subsection{Pour le filtre passe-haut}

\subsubection{Equation de la droite horizontale}

Nous savons que la droite a une valeur initiale de 0.75 V et donc \[y=0.75\]

\subsubsection{Equation de la droite diagonale}

Nous savons que l'équation de la droite est de type $y=a*x+b$
\\
Mais pour cette situation-ci, nous allons utiliser une base logarithmique pour la pente.  En effet, les différentes fréquences utilisées sont tellement éloignées les unes des autres que le graphique serait gigantesque et la pente diagonale serait en fait une courbe.  Ce qui donne: $y=a*\log{x}+b$


Voici 3 résultats mesurés en laboratoire.
\bigbreak
\\
\begin{tabular}{|c|c|c|}
\hline
V_c & f & log{ f} \\
\hline
127 & 0.4 & 2.1\\
\hline
191 & 0.5 & 2.3\\
\hline
356 & 0.6 & 2.6 \\
\hline
\end{tabular}

\bigbreak
Des maintenant les fréquences sont exprimées en base logarithmique et nous obtenons la matrice suivante:
\bigbreak
$$
\begin{pmatrix}  
 2.1 & 1\\
 2.3 & 1 \\
 2.6 & 1 
\end{pmatrix}
\begin{pmatrix}  
a\\
b
\end{pmatrix}
=
\begin{pmatrix}  
0.4\\
0.5\\
0.6
\end{pmatrix}
$$
\bigbreak

Ce qui nous donne les vecteurs suivants:

\[e_1=( \frac{1}{\sqrt[]{3}} \frac{1}{\sqrt[]{3}} \frac{1}{\sqrt[]{3}})\]

\\ et

\\
\[e_2=( -0.6, 0, 0.8)\]

\bigbreak
Ce qui nous donne une projection de 
$$
\begin{pmatrix}  
0.36\\
0.5\\
0.69
\end{pmatrix}$$
$$

\bigbreak
Nous en déduisons la valeur des coefficients a et b:  
\[ a =0.7 \]
\[ b= -1.11 \]

$$\fbox{y= -1.96 \timeslog{x} +9.84}$$

\bigbreak
Pour trouver la fréquence d'intersection entre les deux droites $$y=0.75$$ et $$y= 0.7 \times log{x} -1.11$$ nous égalisons les y et nous trouvons $$\fbox{x=439.4 Hz$$} 

\end{document}
