\documentclass{article}

\usepackage[latin1]{inputenc}
\usepackage[T1]{fontenc}
\usepackage[francais]{babel}

\begin{document}

\section{Dimmensionnement du haut-parleur}

Après avoir réalisé quelques recherches sur les haut-parleurs, nous avons pu imaginer le dispositif idéal à réaliser. En tenant compte des différentes contraintes qui nous étaient imposées, voici les différents choix que nous avons effectués:

\paragraph{Pour le caisson}
Nous devions pouvoir faire varier les fréquences (voir Cahier des Charges), ce qui signifie que nous ne pouvions pas faire un caisson trop petit. Nous avons finalement opté pour une boîte de 30x30x30 cm, étant donné que ces dimensions avaient eu un très beau résultat lors d'un projet d'une année antérieure.

\paragraph{Pour la membrane}Nous avons également opté pour une membrane de diamètre de 17cm. Nous avons choisi cette valeur afin d'avoir une membrane assez large, pour exploiter le mieux possible la taille du caisson. C'est également un diamètre assez répandu dans le commerce. Nous respectons donc les normes.
La profondeur de la membrane sera de 6 cm, comme pour la plupart des haut-parleurs de 17 cm de diamètre. En ce qui concerne la masse surfacique du matériau utilisé, nous avons opté pour quelque chose d'assez rigide mais pas trop, afin d'éviter les difficultés de pliage.


~\\\textbf{Tableau récapitulatif:}
\\
\begin{table*} [h]
	
		\begin{tabular}{|l|c|}
		\hline
		   \textbf{Caractéristique}
			 & \textbf{Justification} \\
		\hline
			Volume du caisson: & Possibilité de faire varier les fréquences. & 30x30x30cm & \\
		\hline
			Diamètre de membrane: & Avoir une membrane assez large,  & 17cm & pour exploiter le mieux possible la taille du caisson. \\
		\hline
			Profondeur de la membrane: & Déterminé en fonction du diamètre de la membrane. & 6cm &\\
		\hline
			Masse surfacique du matériau:  & Facilité de pliage et solidité. & 200 g/$m^2$ &\\
		\hline
		\end{tabular}
		
\end{table*}


\end{document}
