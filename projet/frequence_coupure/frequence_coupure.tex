\documentclass{article}

\usepackage[utf8]{inputenc}
\usepackage[T1]{fontenc}      
\usepackage[francais]{babel}
\usepackage{graphicx}
\usepackage{circuitikz}
\usepackage[squaren, Gray]{SIunits}
\usepackage{sistyle}
\usepackage[autolanguage]{numprint}
\usepackage{pgfplots}
\pgfplotsset{compat=1.9}
\usepackage{amsmath,amssymb,array}
\usepackage[top=2.5cm,bottom=2.5cm,right=2.5cm,left=2.5cm]{geometry}
\usepackage{url} 
\usepackage{tabularx}
\DeclareMathOperator{\dist}{d}
\newenvironment{abstract-fr}
{
	\begin{center}
		\textbf{Résumé} \\[0.5cm]
	\end{center}
}
{}

\newenvironment{abstract-en}
{
	\begin{center}
		\textbf{Summary} \\[0.5cm]
	\end{center}
}
{}
% New command pour la modélisation mécanique, tri à effectuer
\newcommand\fv[1]{{\bf #1}} % free vector
\newcommand\fvd[1]{\dot{\bf #1}} % free vector derivated
\newcommand\fvdd[1]{\ddot{\bf #1}} % free vector derivated
\newcommand\fvr[1]{\mathring{\bf #1}} % free vector relatively derivated
\newcommand\fvrr[1]{\overset{\circ\circ}{\bf #1}} % free vector relatively derivated
\newcommand\uv[1]{{\bf\hat{ #1}}} % unit vector
\newcommand\ui{{\bf\hat{I}}} % unit vector I
\newcommand\uj{{\bf\hat{J}}} % unit vector J
\newcommand\uk{{\bf\hat{K}}} % unit vector K
\newcommand\wrt[2]{\ensuremath{\tensor*[_{ #1}]{ #2}{}}} % With Respect To
\newcommand\wtr[3]{\ensuremath{\tensor*[_{ #1}]{ #2}{^{ #3}}}} % With Two Respect
\newcommand\omegaf{{\bm \omega}}
\newcommand\omegafr{\mathring{\bm \omega}}
\newcommand\omegafd{\dot{\bm \omega}}
\newcommand\omegaft{\tilde{\bm \omega}}
\newcommand\omegaftr{\mathring{\tilde{\bm \omega}}}
\newcommand\omegat{\tilde{\omega}}
\newcommand\omegatd{\tilde{\dot{\omega}}}
\newcommand\ine{{\bf I}}
\newcommand\st{{\bf L}}
\newcommand\pst{{\bf M}}
\newcommand\lm{{\bf N}}
\newcommand\am{{\bf H}}
\newcommand\amd{\dot{\am}}
\newcommand\fo{{\bf F}}
\newcommand\po{\mathcal{P}}
\newcommand\xg{\ensuremath{\fv{R}}}
\newcommand\xgd{\ensuremath{\fvd{R}}}
\newcommand\xgdd{\ensuremath{\fvdd{R}}}
\newcommand\dvec[1]{\dot{\vec{ #1}}}
\newcommand\ddvec[1]{\ddot{\vec{ #1}}}
\newcommand\qp{\dot{q}}
\newcommand\dqp{\Delta \dot{q}}
\usepackage{url} 
\usepackage{hyperref}
\hypersetup{
    colorlinks,
    citecolor=black,
    filecolor=black,
    linkcolor=black,
    urlcolor=black
}

\begin{document}
\section{Approximation de la fréquence de coupure}

Cette section a pour but d'expliquer notre démarche pour l'approximation de la fréquence de coupure dans un circuit passe-bas et passe-haut. 

Définissons tout d'abord ce qu'est la fréquence de coupure: c'est la fréquence à partir de laquelle la tension dans un circuit passe-bas (resp. passe-haut) commence à diminuer (resp. se stabiliser) lorsque la fréquence augmente.
Nous devions donc trouver l'équation de deux droites: une droite horizontale, et une droite oblique. L'intersection de ces droites est appelée "fréquence de coupure".

Pour ce faire, nous avons tout d'abord procédé à une expérience en laboratoire. 
Celle-ci consistait à mesurer la tension de sortie en fonction de la fréquence du signal, et ce dans chacun des circuits considérés. 
Nous avons donc procédé aux mesures pour le circuit passe-bas, ainsi que pour le passe-haut, et cela nous a ensuite permis de modéliser notre problème, et de trouver une fréquence de coupure expérimentale.

\subsection{Pour le filtre passe-bas}

\begin{figure}[h]
   \centering
   \begin{document}

\setlength{\unitlength}{1cm}
\begin{picture}(6,6)(-1,-1)
% axe x
   \put(0,-1){\vector(0,1){6}}
   \multiput(-1,-0.2)(1,0){6}{\line(0,1){0.4}} % traits de graduation

% axe y
   \put(-1,0){\vector(1,0){6}}
   \multiput(-0.2,-1)(0,1){6}{\line(1,0){0.4}} % traits de graduation

% segment
\put(0,4){\line(1,0){2}}
 \put(2,4){\line(1,-2){2}}

\end{picture}

\end{document}

  \caption{\label{premierebissectrice}Tension en fonction de la fréquence dans un filtre passe-bas}
\end{figure}

\subsubsection{Equation de la droite horizontale} % A refaire avec la méthode d'approximation
Expérimentalement, nous obtenons une droite horizontale d'une valeur initiale de $\unit{2.5}{\volt}$ et donc \[y=2.5\]

\subsubsection{Equation de la droite diagonale}

Avec les mesures effectuées en laboratoires, nous n'obtenons non pas une droite mais bien une exponentielle. Pour faciliter le calcul de l'intersection de l'exponentielle et de la droite, nous passons donc en repère semi-logarithmique. En procédant de la sorte, l'exponentielle se transforme en droite, et le calcul devient plus facile.

Nous savons que l'équation d'une droite dans un repère cartésien est de type $y=ax+b$, avec $a$ la pente et $b$ l'ordonnée à l'origine.

Mais ici nous ne sommes plus dans un repère cartésien mais bien dans un répère semi-logarithmique (en $x$). L'équation de la droite devient alors : $y=a\log{x}+b$

\paragraph{Mesures en laboratoires}

Voici 3 résultats choisi (de manière cohérente) parmi toutes les mesures effectuées en laboratoire, où $V_c$ est la tension de sortie et $f$ la fréquence :

\begin{center}
	\begin{tabular}{|c|c|c|}
		\hline
		$V_c$ & $f$ & $\log{f}$ \\
		\hline
		1.7 & 16000 & 4.204 \\
		\hline
		1.55 & 18000 & 4.255 \\
		\hline
		1.45 & 20000 & 4.301 \\
		\hline
	\end{tabular}
\end{center}

Dès maintenant les fréquences sont exprimées en base logarithmique.
Nous écrivons maintenant un système ayant pour inconnues la pente ($a$) et l'ordonnée à l'origne ($b$) de notre droite inconnue. 
Nous avons trois équations à deux inconnues, et le systène n'admet pas de solution. 
Cela n'est pas étonnant, étant donné que les résultats expérimentaux ne sont jamais très précis.

Voici le système sous forme matricielle :

$$A \cdot \vec{x} = \vec{b}$$

\begin{center}
	\begin{array}{rcel}
		$$
		\begin{pmatrix}  
			4.204 & 1\\
			4.255 & 1 \\
			4.301 & 1 
		\end{pmatrix} &

		\begin{pmatrix}  
			a\\
			b
		\end{pmatrix} &

		= &

		\begin{pmatrix}  
			1.7\\
			1.55\\
			1.45
		\end{pmatrix}
		$$
	\end{array}
\end{center}	
	
Le système n'admet pas de solution car $\vec{b}$ n'appartient pas à l'espace des colonnes de $A$. Nous allons donc projeter $\vec{b}$ sur l'espace des colonnes de $A$ afin d'obtenir une solution approchée du système.

Soit $f_1$, $f_2$ les colonnes de $A$, et donc les éléments de la base de l'espace des colonnes de $A$.
Pour ce faire, trouvons une base orthonormée $(e_1, e_2)$ de l'espace colonnes de la matrice en utilisant
la méthode de Gram-Schmidt :

$$e_1 = \frac{f_1}{||f_1||} = (\frac{1}{\sqrt{3}} ; \frac{1}{\sqrt{3}} ; \frac{1}{\sqrt{3}})$$

$$e_2 = \frac{f_2 - (e_1|f_2)}{||f_2 - (e_1|f_2)||} = (-0.684 ; 0.03 ; 0.729)$$


Nous sommes maintenant en mesure de trouver une projection du vecteur contenant nos données expérimentales peu précises :
nous projetons les vecteurs grâce à la formule de la projection :

%	Peut-être parler ici du fait qu'on projete plutôt sur A orthogonal pour gagner du temps en calcul ? 

$$
\vec{b'}
=
(b|e_1) \cdot e_1 + (b|e_2) \cdot e_2
=
\begin{pmatrix}  
1.6\\
1.5\\
1.5
\end{pmatrix}$$
$$

On peut alors réecrire le système comme cela :

$$
\begin{pmatrix}  
 4.204 & 1\\
 4.255 & 1 \\
 4.301 & 1 
\end{pmatrix}
\begin{pmatrix}  
a\\
b
\end{pmatrix}
=
\begin{pmatrix}  
1.607\\
1.565\\
1.524
\end{pmatrix}
$$


Nous avons donc $a\times \log{x} + b = y$ \Rightarrow 
$$
y'=
\begin{pmatrix}  
1.607\\
1.565\\
1.524
\end{pmatrix}
$$
\\

Nous en déduisons la valeur des coefficients $a$ et $b$ :  

$$a = -1.96$$
$$b= 9.84$$

$$\fbox{y= -1.96 \timeslog{x} +9.84}$$

Pour trouver la fréquence d'intersection entre les deux droites $$y=2.5$$ et $$y= -1.96 \times log{x} +9.84$$ nous résolvons le système, et nous trouvons : 

$$\fbox{x = \unit{5557.7}{\hertz}}$$ 

Cela nous semble correct car en théorie nous devions arriver à une valeur $f$ telle que : 

$$f=\frac{1}{2\pi RC}$$

avec $R= 7.5 + 50 = \unit{57.5}{\ohm}$ et $C= \unit{470 \cdot 10^{-9}}{\farad}$, la valeur théorique de la fréquence de coupure est donc :

$$f= \unit{5889.2}{\hertz}$$

\subsection{Pour le filtre passe-haut}

\begin{figure}[h]
		\centering
   \begin{document}

\setlength{\unitlength}{1cm}
\begin{picture}(6,6)(-1,-1)
% axe x
   \put(0,-1){\vector(0,1){6}}
   \multiput(-1,-0.2)(1,0){6}{\line(0,1){0.4}} % traits de graduation

% axe y
   \put(-1,0){\vector(1,0){6}}
   \multiput(-0.2,-1)(0,1){6}{\line(1,0){0.4}} % traits de graduation

% segment
\put(2,4){\line(1,0){2}}
 \put(0,0){\line(1,2){2}}

\end{picture}

\end{document}

		\caption{\label{premierebissectrice}Tension en fonction de la fréquence dans un filtre passe-haut}
\end{figure}

\subsubection{Equation de la droite horizontale}

% Pareil ici, normalement à refaire avec la même méthode d'approximation si on veut que ce soit parfait

Nous savons que la droite a une valeur initiale de $\unit{0.75}{\volt}$ et donc $y = 0.75$.

\subsubsection{Equation de la droite diagonale}

Nous allons encore utiliser une base logarithmique pour la pente, pour les mêmes raison explicitées pour le filtre passe-bas. 
L'équation de la droite est donc de la forme : $y=a\log{x}+b$.

Voici 3 résultats choisis parmis les mesures effectuées en laboratoire :

\begin{center}
	\begin{tabular}{|c|c|c|}
		\hline
		$V_c$ & $f$ & $\log{f}$ \\
		\hline
		127 & 0.4 & 2.1\\
		\hline
		191 & 0.5 & 2.3\\
		\hline
		356 & 0.6 & 2.6 \\
		\hline
	\end{tabular}
\end{center}

De manière analogue que pour le premier filtre, nous formons le système suivant :

\begin{center}
	\begin{array}{rcel}
		$$
		\begin{pmatrix}  
			 2.1 & 1\\
			 2.3 & 1 \\
			 2.6 & 1 
		\end{pmatrix}&

		\begin{pmatrix}  
			a\\
			b
		\end{pmatrix}&

		=&

		\begin{pmatrix}  
			0.4\\
			0.5\\
			0.6
		\end{pmatrix}
		$$
	\end{array}
\end{center}


Transformons maintenant la base quelconque $(f_1, f_2)$ de l'espace des
colonnes de $A$ en une base orthonormée $(e_1, e_2)$ afin de pouvoir 
calculer la projection orthogonale ensuite :

$$e_1 = \frac{f_1}{||f_1||} = (\frac{1}{\sqrt{3}} ; \frac{1}{\sqrt{3}} ; \frac{1}{\sqrt{3}})$$

$$e_2 = \frac{f_2 - (e_1|f_2)}{||f_2 - (e_1|f_2)||} = (-0.6, 0, 0.8)$$

Et nous obtenons comme vecteur projeté :

$$
\begin{pmatrix}  
	0.36\\
	0.5\\
	0.69
\end{pmatrix}
$$

On peut alors réecrire le système comme ceci :

$$
 \begin{pmatrix}  
  2.1 & 1\\
  2.3 & 1 \\
  2.6 & 1 
 \end{pmatrix}
 \begin{pmatrix}  
 a\\
 b
 \end{pmatrix}
 =
 \begin{pmatrix}  
 0.36\\
 0.5\\
 0.69
 \end{pmatrix}
$$

Nous avons donc $a \times \log{x} + b = y$ \Rightarrow 

$$
y'=
\begin{pmatrix}  
0.36\\
0.5\\
0.69
\end{pmatrix}
$$

Nous en déduisons la valeur des coefficients $a$ et $b$ :  

$$a =0.7$$

$$b= -1.11$$

L'équation de la droite est alors :

$$\fbox{y=  0.7 \log{x} - 1.11}$$

Pour trouver l'intersection entre les deux droites $$y=0.75$$ et $$y= 0.7 \times log{x} -1.11$$ nous résolvons le système, et nous trouvons: 

$$\fbox{x = \unit{439.4}{\hertz}}$$ 

% Just here to fix rapport_prejury.tex
\end{document}
