\documentclass[pdf]{beamer}
\usepackage[utf8]{inputenc}
\usepackage{graphicx}
\usepackage[T1]{fontenc}      
\usepackage[francais]{babel}
\usepackage{graphicx}
\usepackage[squaren, Gray]{SIunits}
\usepackage{sistyle}
\usepackage[autolanguage]{numprint}
\usepackage{amsmath,amssymb,array}

\usetheme{warsaw}
\mode<presentation>{}

\title{APP2 - Mécanique}
\author{Groupe 115.3}
\date{\today}

\begin{document}

% =======================================================================
\begin{frame}
	\titlepage
\end{frame}
% =======================================================================
\begin{frame}
	\frametitle{2.1 - Mesurer l'inertie d'un rotor}
	
	Pour mesurer l'inertie de notre rotor, nous avons envisagé une expérience
	qui consiste à accrocher notre rotor au bout d'un fil et à le faire osciller.
	Il est alors possible, à partir de la période des oscillations que l'on peut
	facilement mesurer, de retrouver l'inertie recherchée. 
	
	\begin{center}
		\includegraphics[scale=0.5]{experience.png}
	\end{center}
\end{frame}
% =======================================================================
\begin{frame}
	\frametitle{2.1 - Mesurer l'inertie d'un rotor (suite)}
	En résolvant l'équation du mouvement du pendule, on trouve :
	
	$$I^O_3 = \frac{(L+R)mg}{\omega^2}$$	
	
	Où $\omega = \frac{2\pi}{T}$, $T$ est la période mesurée, $L$ la longueur du fil, 
	$R$ le rayon du rotor, $m$ sa masse et $O$ le point d'attache du fil. On peut
	ensuite retrouver $I^G_3$ par \textit{Steiner}.
	
\end{frame}
% =======================================================================
\begin{frame}
	\frametitle{2.2 - Estimer l'inertie d'un camion}	

	Pour estimer l'inertie d'un camion, nous avons suivi les étapes suivantes :
	
	\begin{enumerate}
		\item Décomposition du camion en 3 corps : le remorque (que l'on considère
		pleine), la cabine (que l'on considère creuse) et le moteur (que l'on considère
		plein également). Les trois corps sont supposés être parallélépipédique ;
		\item Calcul du tenseur d'inertie central de chaque corps séparément ;
		\item Calcul du centre de gravité du camion ;
		\item Utilisation de \textit{Steiner} pour exprimer les trois tenseurs d'inertie
		par rapport au centre de gravité du camion :
		\item Addition des tenseurs d'inertie ;
	\end{enumerate}
\end{frame}
% =======================================================================	
\begin{frame}
	\frametitle{Axes principaux d'inerties}
	Pour un camion avec une remorque de longueur totale $\SI{15}{\meter}$
	et d'une masse totale de $\SI{45000}{\kilogram}$,
	les axes principaux d'inerties sont donc :
	
	$$I_1 = \SI{483252}{\kilogram\meter\squared}$$
	$$I_2 = \SI{357667}{\kilogram\meter\squared}$$
	$$I_3 = \SI{748274}{\kilogram\meter\squared}$$
	
	Avec $\hat{I}_2$ parallèle au camion et $\hat{I}_3$ vertical.
\end{frame}
%========================================================================

\end{document}