\documentclass[pdf]{beamer}
\usepackage[utf8]{inputenc}
\usepackage{graphicx}
\usepackage[T1]{fontenc}      
\usepackage[francais]{babel}
\usepackage{graphicx}
\usepackage[squaren, Gray]{SIunits}
\usepackage{sistyle}
\usepackage[autolanguage]{numprint}
\usepackage{amsmath,amssymb,array}

\usetheme{warsaw}
\mode<presentation>{}

\title{APP2 - Mécanique}
\author{Groupe 115.3}
\date{\today}

\begin{document}

% =======================================================================
\begin{frame}
	\titlepage
\end{frame}

% =======================================================================
\begin{frame}
	\frametitle{APP2.1 - Mesurer l'inertie d'un rotor}
	
	Pour mesurer l'inertie de notre rotor, nous avons envisagé une expérience
	qui consiste à accrocher notre rotor au bout d'un fil et à le faire osciller.
	Il est alors possible, à partir de la période des oscillations que l'on peut
	facilement mesurer, de retrouver l'inertie recherchée.
	
\end{frame}

% =======================================================================
\begin{frame}
	\frametitle{APP2.2 - Estimer l'inertie d'un camion}	

	Pour estimer l'inertie d'un camion, nous avons suivi le cheminement suivant :
	
	\begin{enumerate}
		\item Décomposition du camion en 3 corps : le remorque (que l'on considère
		pleine), la cabine (que l'on considère creuse) et le moteur (que l'on considère
		plein également). Les trois corps sont supposés être parallélipèdique ;
		\item Calcul du tenseur d'inertie central de chaque corps ;
		\item Calcul du centre de gravité du camion ;
		\item Utilisation de Steiner pour ramener les trois tenseurs d'inertie
		au centre de gravité du camion :
		\item Addition des tenseurs d'inertie ;
	\end{enumerate}
	
\end{frame}

\end{document}