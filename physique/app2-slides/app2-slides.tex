\documentclass[pdf]{beamer}
\usepackage[utf8]{inputenc}
\usepackage{graphicx}

\usetheme{warsaw}
\mode<presentation>{}

\title{Physique}
\subtitle{APP 2 : Le magnétisme dans et en dehors de la matière}
\author{Groupe 115.3}

\begin{document}

\begin{frame}
	\titlepage
\end{frame}

\begin{frame}{Les formules}
	\underline{Loi d'Ampère} \\
	$$\oint \vec{H}\cdot\vec{dl}\cong H_m \cdot L + H_e \cdot e = N \cdot I$$ \\
	\center{Quand nous avons} $\ H_e \cong H_i$
	$$\oint \vec{H} \cdot \vec{dl} \cong H \cdot l = N \cdot I$$
	Dans le noyau $B_i=\mu_0 \cdot \mu_r \cdot H_i$ \\
	A la surface latérale $B_e = \mu_0 \cdot H_e$ \\
\end{frame}

\begin{frame}{Dans un circuit magnétique}
	\center{Dans un électroaimant sans entrefer tout le champ magnétique B se retrouve dans le matériau.} \\
	\center{Dans un électroaimant avec entrefer tout le champ magnétisant H se retrouve dans l'entrefer.} \\
\end{frame}

\begin{frame}{Première situation}
	\center{Il n'y a aucun entrefer.} \\
	\center{La champ H et le champ B sont identiques partout.}
	\includegraphics[scale=0.5]{Circuitalpha.png}
	\center{\underline{Nos solutions :}}
	\center{Nous obtenons 188 spires et $H = 497,36 \frac{A}{m}$ pour} \\
	$B=1T \ I=0.5A \ \mu_r=1600$
\end{frame}

\begin{frame}{Deuxième situation}
	\center{Il y a un entrefer de 1cm.}
	\center{Le champ H est uniforme}
	\includegraphics[scale=0.5]{Circuitbeta}
	\center{\underline{Nos solutions :}}
	\center{Nous obtenons $H_e = 94000 \frac{H}{m}$  \ et $H_m=0$ pour} \\
	$B_e=0.06T$ \ $e=1mm$ 
\end{frame}

\begin{frame}{Troisième situation}
	\center{L'entrefer est maintenant de 11mm.}
	\includegraphics[scale=0.5]{Circuitgamma}
	\center{\underline{Nos solutions:}}
	\center{Pour un champ de $B=5,3\cdot10^{-3}T$} \\ 
	\center{Il faut un nombre $\cong 35000$ spires.}
\end{frame}

\begin{frame}
	\center{\Huge{MERCI DE VOTRE ATTENTION !}}
\end{frame}

\end{document}