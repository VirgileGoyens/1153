\documentclass[pdf]{beamer}
\usepackage[utf8]{inputenc}
\usepackage{graphicx}
%\usepackage[T1]{fontenc}
%\usepackage[francais]{babel}
\usetheme{warsaw}
\mode<presentation>{}
%% preamble
\title{Physique}
\subtitle{App 2: Le magnétisme dans et en dehors de la matière}
\author{Groupe 115.3}
\begin{document}
%% title frame
\begin{frame}
\titlepage
\end{frame}
%% normal frame
\begin{frame}{Les formules}
\underline{Loi d'Ampère} \\
$$\oint \vec{H}\cdot\vec{dl}\cong H_m * L + H_e * e = N*I$$ \\
\center{Quand nous avons} $\ H_e \cong H_i$
$$\oint \vec{H} \cdot \vec{dl} \cong H*l = N*I$$
Dans le noyau $ B_i=\mu_0*\mu_r*H_i$ \\
A la surface latérale $B_e = \mu_0*H_e$ \\
\end{frame}
\begin{frame}{Dans un circuit magnétique}
\center{Dans un électroaimant sans entrefer tout le champ magnétique B se retrouve dans le matériau.} \\
\center{Dans un électroaimant avec entrefer tout le champ magnétisant H se retrouve dans l'entrefer.} \\
\end{frame}
\begin{frame}{Première situation}
\center{Il n'y a aucun entrefer.} \\
\center{La champ H et le champ B est identique partout.}
\includegraphics[scale=0.5]{Circuitalpha.png}
\center{\underline{Nos solutions:}}
\center{Nous obtenons 188 spires et $H = 497,36 \frac{A}{m}$ pour} \\
$ B=1T \ I=0.5A \ \mu_r=1600 $
\end{frame}
\begin{frame}{Deuxième situation}
\center{Il y a un entrefer de 1cm}
\center{Le champ H est uniforme}
\includegraphics[scale=0.5]{Circuitbeta}
\center{\underline{Nos solutions:}}
\center{Nous obtenons $H_e=94000\frac{H}{m}$  \ et $H_m=0$ pour} \\
$B_e=0.06T$ \ $e=1mm$ 
\end{frame}
\begin{frame}{Fils de puterie}
\center{\Huge{Ca fart les gars?}}

\end{frame}
\end{document}
