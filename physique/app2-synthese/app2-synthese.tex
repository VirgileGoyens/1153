\documentclass{report}

\usepackage[latin1]{inputenc}
\usepackage[T1]{fontenc}
\usepackage[francais]{babel}

\title{App 2  Physique}
\author{Groupe 11.53}
\date{fevrier 2014}
\begin{document}
 
\maketitle

\chapter{La plaque est collée contre le "E"}
\section{Les lignes de champs magnétique}

\section{Determiner le nombre de spires}

Hypothèses: 
\begin{enumerate}
\item Il n'y a pas de perte de champ magnétique, tout le champ reste dans la pièce de métal.
\end{enumerate}

Nous savons, d'après l'APP 1 que, \[B = 1T\] 

Avec la formule \[B = µ_{0} \times µ_{r} \times Hi\]  En connaissant la valeur des constantes  \[µ_{0} = 4\pi \times 10^{-7}\] et \[µ_{r} = 1600 \] Nous pouvons trouver Hi.

\[ Hi = 497.39 \frac{A}{m}\]

Ensuite, nous avons utilisé une autre formule pour pouvoir déterminer le nombre de spires dont nous aurons besoins.

\[H_{i} \times l = N I\]

Comme nous connaissons l'intensité du courant, le Hi que nous venons de calculer, il reste à calculer la longeur des lignes de champs magnetique.  Vu que nous connaissons les dimensions de la pièce, nous pouvons calculer la longeur moyenne d'une ligne de champs.  

\medbreak longeur d'un long coté: 78-11 = 67 mm
\medbreak longeur d'un petit coté: (3x11)-5.5 = 27.5 mm

\bigbreak Nous avons donc une longeur moyenne de 189 mm.

En insérant cette dernière donnée dans la formule, nous trouvons le nombre de spires. 
\[\frac{497.36 \times 0.189}{0.5} = 188 spires\]

Nous avons donc \fbox{188 spires}
\chapter{La plaque est légèrement décollée du "E"}
\section{Les lignes de champs magnétique}
\section{Calcul du champ magnétique}

Hypothèse:
\begin{enumerate}
\item Nous considerons le champs magnétisant à l'interieur du materiau comme étant nul.  $H_{m} = 0$ car $µ_{r}=0$
\item Nous prenons une distance de 1 mm entre le "E" et la plaque de métal.
\item Encore une fois, il n'y a pas de perte de champ magnétique, tout ce qui sort rentre dans le métal.
\end{enumerate}

Cette fois-çi, les formules sont quelque peu différentes.
\\
Nous avons \[H_{m}\times L + H_{e} \times e = NI\] mais comme dans l'hypothèse nous avons $H_{m}=0$ la formule devient 
\[H_{e} \times e = NI\]  
Comme nous avons le nombre de spires ( section précédente), l'intensité et la distance d'entrefer, nous pouvons trouver H_{e}.

\emph{Nous prenons une distance de 2e dans la formule car le champ sort une fois et rentre une fois, il y a donc 2 entrefers}

\[H_{e} \times 2 \times e = N I\]
\[H_{e}= 47000 \frac{A}{m}\]

L'autre formule nous dit que dans l'entrefer:\[ B_{e} = µ_{0} \times H_{e}\]
\[B_{e} = 0.059 T\]

Nous avons donc \fbox{0.059 T}

\chapter{La plaque est supprimée}
\section{Les lignes de champ magnétique}
\section{ Estimation du champ magnétique dans l'entrefer}

Hypothèse:
\begin{enumerate}
\item Il n'y a pas de perte de champs magnétique.
\item Les lignes de champs sortant de la barre du milieu du "E" rentre dans une des des deux barres extérieure.
\item Le champ magnétisant à l'intérieur du materiau vaut 0, car µ_{r} =0.
\end{enumerate}

Les formules sont les mêmes que pour la section précédente.

\[H_{e} \times 2 \times e = NI\]
\[H_{e} = \frac{188 \times 0.5}{2 \times 0.011} = 4272.72 \frac{A}{m}\]

Et donc nous pouvons trouver \[ B_{e} = µ_{0} \times H_{e}\]
\[B_{e} = 5.369 \times 10^{-3} T\]

Nous avons donc un champ magnétique d'entrefer de \fbox{$5.369 \times 10^{-3} T$}

\section{Réajuster le nombre de spires nécessaires}

Si nous voulons un champ magnétique d'un Tesla, on aura besoin de 35014 spires.

\end{document}
