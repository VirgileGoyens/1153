\documentclass{article}

\usepackage[latin1]{inputenc}
\usepackage[T1]{fontenc}      
\usepackage[francais]{babel}
\usepackage{chemist}

\title{Chimie-Physique I - Devoir 1}
\author{Groupe 115.3}
\date{\today}

\begin{document}

\maketitle

\begin{enumerate}
	\item 
		L'\'equation de la r\'eaction est donn\'ee par :
		\begin{chemmath}
			3CuSO_4 \cdot 5H_2O (s) + 2PH_3 (g) \longrightarrow Cu_3P_2 (s) + 3H_2SO_4 (aq) + 15H_2O (l)
		\end{chemmath}
	\item 
		\begin{itemize}
			\item \chemform{CuSo_4 \cdot 5H_2O} : sulfate de cuivre (II) pentahydrat\'e ;
			\item \chemform{PH_3} : phosphine ;
			\item \chemform{Cu_3P_2} : phosphure de cuivre (II) ;
			\item \chemform{H_2SO_4} : acide sulfurique ;
			\item \chemform{H_2O} : eau (monoxyde de dihydrog\`ene).
		\end{itemize}
	\item
		La masse molaire du sulfate de cuivre pentahydrat\'e est donn\'ee, elle vaut : 249.68 g/mol. On connait 			\'egalement la masse du sulfate de cuivre pentahydrat\'e, elle vaut : 110 g. On a donc le nombre de mole, donn\'e par $\frac{110}{249.68} = 0.44$ mol.
		
		La masse molaire de la phosphine vaut quant \`a elle 34 g/mol. La masse pr\'esente de la phosphine est de 4.94 g \`a 85\%. On trouve alors 0.126 mol.
		
		Le r\'eactif limitant est donc la phosphine, \'etant donn\'e que 0.44 mol \chemform{CuSo_4 \cdot 5H_2O}/3 > 0.126 mol \chemform{PH_3}/2. 
		
	\item
		Le rendement de la r\'eaction est de 6.31\% en masse. Avec un rendement de 100\%, on obtiendrait 0.06175 mol. Calculons la masse correspondante à ce nombre de moles. On a $M = 252.56$ g/mol, on trouve alors $m = 15.59558$g. Multipli\'e par le rendement de 6.31\%, on obtient : $15.59558 \cdot 0.0631 = 0.984$g.
\end{enumerate}

\end{document}
