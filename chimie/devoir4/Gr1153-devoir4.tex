\documentclass{article}

\usepackage[utf8]{inputenc}
\usepackage[T1]{fontenc}      
\usepackage[francais]{babel}
\usepackage{chemist}
\usepackage{chemfig} 
\usepackage{lewis}

\newcommand\exercice[1]{%
\paragraph{#1}%
~\par
~\par}

\newcommand\chemfigc[1]{
\vspace{0.5cm}
\begin{center}\chemfig{#1}\end{center}
\vspace{0.5cm}}

\title{Chimie-Physique I - Devoir 4}
\author{Groupe 115.3}
\date{\today}

\begin{document}

\maketitle

\exercice{Exercice 2.88}

\begin{enumerate}\renewcommand{\theenumi}{\alph{enumi}}
	\item La structure de Lewis du \chemform{NHCH_2} est :
	
	\chemfigc{H-\lewis{2, N}=C(-[:90]H)(-H)}
	
	On a donc une double liaison entre l'atome de carbone et
	l'atome d'azote.
	
	\item La structure de Lewis du \chemform{NH_2CH_3} est :
	
	\chemfigc{H-\lewis{6, N}(-[:90]H)-C(-[:90]H)(-[:-90]H)-H}
	
	On a donc une simple liaison entre l'atome de carbone et 
	l'atome d'azote.
	
	\item La structure de Lewis du \chemform{HCN} est :
	
	\chemfigc{H-C~\lewis{0, N}}
	
	On a donc une triple liaision entre l'atome de carbone et
	l'atome d'azote.
\end{enumerate}

\textbf{En conclusion}, le composé ayant la liaison \chemform{C-N}
la plus forte est le \chemform{HCN}.

\exercice{Exercice 5.34}

\begin{enumerate}\renewcommand{\theenumi}{\alph{enumi}}
	\item 
		\begin{itemize}
			\item
				\chemform{CH_3CL} : interactions dipôle permanent -
				dipôle permanent (force de Keesom) et interactions
				dipôle induit - dipôle induit (force de London).
				% Src : http://tinyurl.com/devoir4src1 & http://tinyurl.com/devoir4src3
			\item
				\chemform{CH_4} : interaction dipôle induit - dipôle 
				induit (force de London). 
				% Src : http://fr.wikipedia.org/wiki/Dipôle_induit
			\item
				\chemform{CH_3COOH} : pont hydrogène (entre
				\chemform{O} et \chemform{H}).
				% Src : http://tinyurl.com/devoir4src2
			\end{itemize}
	\item Plus il y a d'interactions intermoléculaires et plus
				elles sont fortes, plus la températeur d'ébullition 
				et de fusion augmente. On a donc : \chemform{CH_4}
				$<$ \chemform{CH_3CL} $<$ \chemform{CH_3COOH}. En effet : 
				ponts H $>$ Keesom $>$ Debye $>$ London. 
\end{enumerate}	

\exercice{Exercice 5.36}

\begin{enumerate}\renewcommand{\theenumi}{\alph{enumi}}
	\item	\chemform{FeS_2} : covalent car failbe $\Delta \chi$ : 0.75
	\item	\chemform{C_8H_{19}} : covalent car faible $\Delta \chi$ : 0.35
	\item	\chemform{BN} : covalent car faible $\Delta \chi$ : 1
	\item	\chemform{CaSO_4} : le \chemform{CaSO_4} résulte de l'
				attraction électrostatique entre le \chemform{Ca^{2+}}
				et le \chemform{SO_4^{2-}}. On a donc une liaison ionique.
	\item	\chemform{Cr_2} : métalique.
\end{enumerate}	
				
\end{document}
