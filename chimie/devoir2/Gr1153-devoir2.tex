\documentclass{article}

\usepackage[utf8]{inputenc}
\usepackage[T1]{fontenc}      
\usepackage[francais]{babel}
\usepackage{chemist} 

\newcommand\exercice[1]{%
\paragraph{#1}%
~\par
~\par}

\title{Chimie-Physique I - Devoir 2}
\author{Groupe 115.3}
\date{\today}

\begin{document}

\maketitle

\exercice{Exerice 1.78, page 52}

\begin{enumerate}
  \item \chemform{Zn} : $1s^22s^22p^63s^23p^64s^23d^{10}$ : 1 électron en moins sur la couche $4s$ ;
  \item \chemform{Cl} : $1s^22s^22p^63s^23p^5$ : 1 électron en moins sur la couche $3p$ ;
  \item \chemform{Al} : $1s^22s^22p^63s^23p^1$ : 1 électron en moins sur la couche $3p$ ;
  \item \chemform{Cu} : $1s^22s^22p^63s^23p^64s^13d^{10}$ : 1 électron en moins sur la couche $4s$.
\end{enumerate}

\exercice{Exercice 1.90, page 52}

\begin{enumerate}
	\item \chemform{Ca^{2+}}, \chemform{Ba^{2+}} : \chemform{Ba^{2+}} car $n = 5$ contre $n = 3$ pour \chemform{Ca^{2+}} ;
	\item \chemform{As^{3-}}, \chemform{Se^{2-}} : \chemform{As^{3-}} car le noyeau de \chemform{Se^2-} est plus attractif.
	\item \chemform{Sn^{2+}}, \chemform{Sn^{4+}} : \chemform{Sn^{2+}} car $n = 5$ contre $n = 4$ pour \chemform{Sn^{4+}}.
\end{enumerate}

\exercice{Exercice 1.124, page 54}

On sait que :

% Formule page 37 dans les slides
$$\Delta E = Ry(\frac{1}{n^2_f} - \frac{1}{n^2_i})$$

où $Ry = 2.178 \cdot 10^{-18} J$.

\begin{enumerate}
	\item Pour qu'un électron passe de l'orbitale $4d$ à $1s$, l'énergie du photon doit être :
				$$\Delta E = E_1 - E_4 = 2.178 \cdot 10^{-18} \cdot (\frac{1}{1^2} - \frac{1}{4^2}) = 2.04 
				\cdot 10^{-18} J$$
	\item	Pour qu'un électron passe de l'orbitale $4d$ à $2p$, l'énergie du photon doit être :
				$$\Delta E = E_1 - E_4 = 2.178 \cdot 10^{-18} \cdot (\frac{1}{2^2} - \frac{1}{4^2}) = 4.088 
				\cdot 10^{-19} J$$
	\item Pour qu'un électron passe de l'orbitale $4d$ à $2s$, l'énergie du photon doit être :
				$$\Delta E = E_1 - E_4 = 2.178 \cdot 10^{-18} \cdot (\frac{1}{2^2} - \frac{1}{4^2}) = 4.088 
				\cdot 10^{-19} J$$
				
				On remarque que l'énergie est la même que pour passer de la couche $4s$ à $2p$ car les couches
				avec le même nombre quantique principal ont la même énergie.
	\item	Par la même formule, on trouve 0. Cela s'explique aussi par le fait que les couches avec le même 
				nombre quantique ont la même énergie.
	\item Il y a plus de lignes de spectres parce qu'il y a plus d'électrons, et donc plus de niveaux d'énergie différents.
\end{enumerate}

\end{document}
