\documentclass{article}

\usepackage[latin1]{inputenc}
\usepackage[T1]{fontenc}      
\usepackage[francais]{babel}
\usepackage{chemist}

\title{Chimie-Physique I - Devoir 2}
\author{Groupe 115.3}
\date{\today}

\begin{document}

\maketitle

Exercices 1.78 et 1.90, page 52 / 1.124 page 54.

\paragraph{1.78}

\begin{enumerate}
  \item \chemform{Zn} : 1 électron en moins sur la couche $4s$ ;
  \item \chemform{Cl}: 1 électron en moins sur la couche $3p$ ;
  \item \chemform{Al}: 1 électron en moins sur la couche $3p$ ;
  \item \chemform{Cu}: 1 électron en moins sur la couche $4s$.
\end{enumerate}

\paragraph{1.90}

\begin{enumerate}
	\item \chemform{Ca^{2+}},\chemform{Ba^2+}: \chemform{Ba^2+} car $n = 5$ contre $n = 3$ pour \chemform{Ca^2+} ;
	\item \chemform{As^{3-}},\chemform{Se^2-}: \chemform{As^{3-}} et \chemform{Se^{2-}} ont le même rayon ionique 
	car leur configuration électronique est identique $1s^2 2s^2 2p^6 3s^2 3p^6 3d^{10}4s^2 4p^6$ ;
	\item \chemform{Sn^{2+}},\chemform{Sn^4+}: \chemform{Sn^2+} car $n = 5$ contre $n = 4$ pour \chemform{Sn^4+} ;
\end{enumerate}

\paragraph{1.124}

%TODO

\end{document}